\chapter{Analýza existujících řešení}\label{ch:analyza-existujicich-reseni-pro-praci-s-formulari}

V době psaní této práce jsem nenalezl žádné řešení pro online spolupráci terapeutů a pacientů/klientů.
V této kapitole se tedy zaměříme na analýzu existujících řešení pro tvorbu formulářů a sběr dat.

\section*{Google Formuláře}\label{sec:google-formulare}

\href{https://www.google.com/forms/about/}{Google Formuláře} je jeden z nejrozšířenějších nástrojů pro tvorbu formulářů.
Formuláře v tomto nástroji lze snadno vytvářet v prohlížeči a následně sdílet pomocí odkazu s plniteli.
Jeden formulář může spravovat více uživatelů.
Výsledky formulářů lze analyzovat pomocí nástroje \href{https://www.google.com/intl/cs/sheets/about/}{Google Tabulky} či exportovat do souboru.
Nevýhodou Google Formulářů je, že neumožňují přizpůsobení vzhledu formulářů.
Další nevýhodou je, že Google Formuláře sbírají data o uživatelích, kteří formulář vyplňují.
Google Formuláře nenabízejí variantu s hostováním na vlastní infrastruktuře.

\section*{Form.io}\label{sec:formio}

\href{https://form.io/}{Form.io} poskytuje veškeré funkce, které jsou vyžadovány zadavatelem.
Form.io umožňuje velkou míru upravení vzhledu a fungování formulářů.
Form.io je používáno mnoha velkými společnostmi jako je Visa či ICANN\@.
Software lze hostovat na vlastní infrastruktuře.
Dle aktuálního \href{https://form.io/pricing}{ceníku} varianta s hostováním na vlastní infrastruktuře však stojí \$900 měsíčně.

\section*{OpnForm}\label{sec:opnform}

\href{https://opnform.com/}{OpnForm} je open-source řešení, které je aktivně vyvíjeno.
Je to poměrně mladý projekt, který byl založen v září roku 2022.
Projekt lze hostovat na vlastní infrastruktuře a zdá se, že splňuje všechny požadavky zadavatele.
Projekt však zatím nemá žádnou dokumentaci.

\section*{Typeform}\label{sec:typeform}

\href{https://www.typeform.com/}{Typeform} je úspěšný nástroj pro tvorbu online formulářů.
Nástroj je velmi vyspělý a poskytuje mnoho funkcí včetně větvení formulářů na základě předchozích odpovědí.
Typeform také nenabízí variantu s hostováním na vlastní infrastruktuře.
