\chapter{Analýza existujících řešení}\label{ch:analyza-existujicich-reseni-pro-praci-s-formulari}

Tato sekce popisuje nedostatky existujících platforem pro práci s formuláři.

\section{Google Formuláře}\label{sec:google-formulare}

Google Formuláře nenabízejí variantu s hostováním na vlastní infrastruktuře.

\section{Formio}\label{sec:formio}

Formio poskytuje veškeré funkce, které jsou vyžadovány zadavatelem.
Dle aktuálního \href{https://form.io/pricing}{ceníku} varianta s hostováním na vlastní infrastruktuře však stojí \$900 měsíčně.
Celý produkt má však open-source jádro, které má licenci Open Software License 3.0, která dovoluje komerční použití.
Nevýhodou je, že použití vlastního autentifikačního poskytovatele je dostupné pouze v placených verzích.

\section{OpnForm}\label{sec:opnform}

OpnForm je open-source řešení, které je aktivně vyvíjeno.
Je to poměrně mladý projekt, který byl založen v září roku 2022.
Projekt zatím nemá žádnou dokumentaci.

\section{Typeform}\label{sec:typeform}

Typeform také nenabízí variantu s hostováním na vlastní infrastruktuře.

\section{Formily}\label{sec:formily}

Formily je velký open-source projekt, který dosáhl poměrně velké popularity.
Za projektem stojí velká komerční firma Alibaba. 
Projekt je aktivně vyvíjen, ale bohužel v době volby technologie nebyla dostupná dokumentace v anglickém jazyce.

\section{Závěr}\label{sec:zaver}

Po zvážení těchto možností jsem se rozhodl použít jádro produktu Formio.
Produkt má spoustu dokumentace, kvalitní API a je stále aktivně vyvíjen.
Projekt je zastřešen komerční firmou, což zvyšuje jeho dlouhodobou udržitelnost.
Některé placené funkce by použití ulehčily, ale nejsou pro tento projekt nutné.
