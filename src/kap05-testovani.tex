\chapter{Testování aplikace}\label{ch:testovani-aplikace}

Serverová část aplikace je testována pomocí unit testů.
Testujeme pouze veřejné rozhraní všech modulů.
Objekt fungující jako proxy databáze byl nahrazen mock objekty.
V testech kontrolujeme, zda-li se volají konkrétní metody na mock objektech s očekávanými parametry.
Toto je doporučený přístup object-relational mapping knihovny Prisma, který je popsán v \href{https://www.prisma.io/docs/guides/testing/unit-testing}{dokumentaci}.
Tento přístup se mi vůbec neosvědčil.
Kontrola volání konkrétních metod vytváří obrovskou závislost na vnitřní implementaci testovaných metod.
Testy jsou velmi těžko udržovatelné a navíc poměrně dlouhé a složité.
Kdybych začínal znovu, tak bych se testoval i privátní metody, čímž bych se alespoň částečně vyhnul nutnosti mockování databáze.
Přestože tento způsob je také silně závislý na vnitřní implementaci testovaného modulu, tak je udržitelnější a jednodušší na implementaci.
Pro testování metod obsahující databázové operace bych zvážil použití in-memory databáze\footnote{In-memory databáze je databáze spoléhající primárně na vnitřní paměť pro ukládání dat~\cite{in-memory-db-definition}}.
