\chapter{Testování aplikace}\label{ch:testovani-aplikace}

Serverová část aplikace je testována pomocí automatizovaných unit testů.
Testujeme pouze veřejné rozhraní všech modulů.
Objekt fungující jako proxy databáze byl nahrazen mock objekty.
V testech kontrolujeme, zda-li se volají konkrétní metody na mock objektech s očekávanými parametry.
Ačkoliv je vyše popsaný způsob doporučeným přístupem dle \href{https://www.prisma.io/docs/guides/testing/unit-testing}{dokumentace object-relational mapping knihovny Prisma}, v našem případě se neosvědčil.

Kontrola volání konkrétních metod vytváří obrovskou závislost na vnitřní implementaci testovaných metod.
Testy jsou velmi těžko udržovatelné a navíc poměrně dlouhé a složité.
Kdybych začínal znovu, tak bych testoval i privátní metody, čímž bych se alespoň částečně vyhnul nutnosti mockování databáze.
Přestože tento způsob je také silně závislý na vnitřní implementaci testovaného modulu, tak je udržitelnější a jednodušší na implementaci oproti použití mocků.
Pro testování metod obsahující databázové operace bych zvážil použití in-memory databáze\footnote{In-memory databáze je databáze spoléhající primárně na vnitřní paměť pro ukládání dat~\cite{in-memory-db-definition}}.

V tabulce~\ref{tab:test-coverage} je zobrazeno pokrytí serverové části automatizovanými unit testy.

\begin{table}[h!]
    \centering
    \begin{tabularx}{\textwidth}{
        | >{\centering\arraybackslash}X
        | >{\centering\arraybackslash}X
        | >{\centering\arraybackslash}X
        | >{\centering\arraybackslash}X |
    }
        \hline
        \textbf{Výrazy} & \textbf{Větve} & \textbf{Funkce} & \textbf{Řádky} \\
        \hline
        90.85 \%        & 72.91 \%       & 100 \%          & 90.85 \%       \\
        \hline
    \end{tabularx}
    \caption{Pokrytí serverové části testy}
    \label{tab:test-coverage}
\end{table}