\chapter{Administrátorská příručka}\label{ch:administratorska-prirucka}

Tato kapitola popisuje, jak spravovat aplikaci z pohledu administrátora.
První jsou popsány kroky pro instalaci a konfiguraci aplikace v sekci~\ref{sec:instalace}.
Následně je popsáno spuštění aplikace v sekci~\ref{sec:spusteni-aplikace}.
V sekci~\ref{sec:tvorba-uzivatelskych-uctu} jsou popsány kroky pro tvorbu uživatelských účtů.
Nakonec jsou popsány přístupy k správcovským rozhraním jednotlivých částí aplikace v sekci~\ref{sec:sprava-aplikace}.

Obsah této kapitoly je k dispozici i v repozitáři projektu v souboru \lstinline{README.md}.

\section{Instalace a konfigurace}\label{sec:instalace}

Pro instalaci aplikace je potřeba mít nainstalovaný \href{https://www.docker.com/}{Docker} a \href{https://git-scm.com/}{Git}.

Celý projekt je zastřešen git repozitářem \href{https://github.com/PatrikTrefil/mental-health-monitoring-platform.git}{mental-health-monitoring-platform}.
Tento repozitář používá \href{https://git-scm.com/book/en/v2/Git-Tools-Submodules}{git submoduly}, jelikož zdrojový kód platformy Form.io je v jiném repozitáři.
Jedná se o repozitář \href{https://github.com/PatrikTrefil/mental-health-monitoring-platform-formio}{mental-health-monitoring-platform-formio}, což je fork repozitáře \href{https://github.com/formio/formio}{formio}, který obsahuje oficiální zdrojový kód platformy Form.io.
Pro instalaci aplikace je potřeba naklonovat repozitář \href{https://github.com/PatrikTrefil/mental-health-monitoring-platform.git}{mental-health-monitoring-platform} včetně submodulů, což lze provést následujícím příkazem:

\begin{lstlisting}[language=bash]
git clone --recurse-submodules \
  https://github.com/PatrikTrefil/mental-health-monitoring-platform.git
\end{lstlisting}

Klonování a inicializaci submodulů lze provést i pomocí následující sekvence příkazů:

\begin{lstlisting}[language=bash]
git clone \
  https://github.com/PatrikTrefil/mental-health-monitoring-platform.git
cd mental-health-monitoring-platform
git submodule init # initialize configuration file
git submodule update # fetch submodule data
\end{lstlisting}

Před spuštěním aplikace je potřeba dodat \lstinline{.env} soubor do kořenu repozitáře pro konfiguraci prostředí.
Syntax konfiguračního souboru je popsána v \href{https://docs.docker.com/compose/environment-variables/env-file/}{dokumentaci Docker compose}.
Ukázkovou konfiguraci najdete v \lstinline{.env.example}.
Zde je seznam proměnných prostředí, které je potřeba nastavit, a jejich význam:

\begin{description}[style=nextline]
    \item[MONGO\_INITDB\_ROOT\_USERNAME] Proměnná obsahuje uživatelské jméno pro přihlášení do MongoDB jako superuživatel (root).
    \item[MONGO\_INITDB\_ROOT\_PASSWORD] Proměnná obsahuje heslo pro do MongoDB jako superuživatel (root).
    \item[FORMIO\_NODE\_CONFIG] Proměnná obsahuje konfigurace Form.io aplikace.
    Hodnota by měla být ve formátu JSON\@.
    Výchozí hodnoty konfigurace jsou dostupné \href{https://github.com/formio/formio/blob/3.5.x/config/default.json}{zde}.
    Není potřeba nastavit všechny atributy konfiguračního objektu, ale pouze ty, které chceme přepsat.
    \item[FORMIO\_ROOT\_EMAIL] Proměnná obsahuje e\babelhyphen{nobreak}mail pro přihlášení do Form.io aplikace jako superuživatel (root).
    \item[FORMIO\_ROOT\_PASSWORD] Proměnná obsahuje heslo pro přihlášení do Form.io aplikace jako superuživatel (root).
    \item[FORMIO\_MONGO\_USER] Proměnná obsahuje uživatelské jméno pro přístup Form.io apliakace do MongoDB\@.
    \item[FORMIO\_MONGO\_PASSWORD] Proměnná obsahuje heslo pro přihlášení uživatele \lstinline{FORMIO_MONGO_USER} do MongoDB\@.
    \item[DOMAIN\_NAME] Proměnná obsahuje doménové jméno, na kterém bude aplikace dostupná (např.\ domena.cz, localhost).
    \item[NEXTAUTH\_SECRET] Proměnná obsahuje klíč pro šifrování autentizačních tokenů.
\end{description}

Také je potřeba nainstalovat závislosti pro Form.io aplikaci:

\begin{lstlisting}[language=bash]
cd src/formio && npm install
\end{lstlisting}

Nyní aplikaci můžeme spustit pomocí návodu v sekci~\ref{sec:spusteni-aplikace}.
Poté je potřeba vytvořit první uživatelský účet, jak je popsáno v sekci~\ref{sec:tvorba-uzivatelskych-uctu}.


\section{Spuštění aplikace}\label{sec:spusteni-aplikace}

Předpokládáme, že aplikace byla nainstalována a konfigurována podle návodu v sekci~\ref{sec:instalace}.
Pro spuštění aplikace v produkčním módu spusťte následující příkaz z kořenového adresáře repozitáře:

\begin{lstlisting}[language=bash]
docker compose up
\end{lstlisting}

Po spuštění příkazu bude aplikace dostupná na \href{http://localhost}{\lstinline{http://localhost}}.

Pokud se jedná o první spuštění aplikace, vytvořte prvního uživatele dle návodu v sekci~\ref{sec:tvorba-uzivatelskych-uctu}.

Pro spuštění aplikace ve vývojovém módu použijte následující příkaz:

\begin{lstlisting}[language=bash]
docker compose --file ./docker-compose.yml \
  --file ./docker-compose.dev.yml \
  up
\end{lstlisting}

Po spuštění příkazu bude aplikace dostupná na \href{http://localhost:8080}{\lstinline{http://localhost:8080}}.
Na jednotlivé služby aplikace se lze připojit i přímo.
Mapování portů je definováno v \lstinline{docker-compose.dev.yml}.


\section{Tvorba uživatelských účtů}\label{sec:tvorba-uzivatelskych-uctu}

Tato sekce popisuje kroky pro vytvoření uživatelských účtů.
Účty uživatelů libovolné role lze vytvořit pomocí webového rozhraní Form.io aplikace.
Po prvním spuštění aplikace je doporučeno vytvořit uživatele s rolí správce dotazníků.
Pro jednoduchost tvorby tohoto uživatele je připraven shell skript, jehož užití je popsáno v sekci~\ref{subsec:tvorba-uctu-pomoci-shell-skriptu}.

\subsection{Tvorba účtu pomocí webového rozhraní}\label{subsec:tvorba-uctu-pomoci-weboveho-rozhrani}

Účet lze vytvořit pomocí webového rozhraní Form.io aplikace, která je dostupná na \lstinline{/formio/} po spuštění aplikace.
Po otevření rozhraní se zobrazí přihlašovací formulář.
Do formuláře vyplníme přihlašovací údaje adminisrátorského účtu, které jsme definovali v konfiguračním souboru při instalaci aplikace (viz sekce~\ref{sec:instalace}).
Po přihlášení je potřeba vyplnit formulář \textit{Správce dotazníků registrace}.
Uživatele \textit{nevytvářejte} skrze modul \textit{Resources}.

\subsection{Tvorba účtu pomocí shell skriptu}\label{subsec:tvorba-uctu-pomoci-shell-skriptu}

Pokud máte k dispozici příkaz \href{https://curl.se/}{\lstinline{curl}}, tak uživatele \textit{s rolí správce dotazníků} můžete vytvořit pomocí skriptu \lstinline{./src/scripts/create_form_manager_user.sh}.
Skript se spustí v interaktivním módu, pokud uživatel neposkytne všechny parametry programu.
Parametry programu lze take předat pomocí argumentů programu.
Parametry programu se odvíjí od konfigurace definované při instalaci aplikace (viz sekce~\ref{sec:instalace}).
Dokumentaci nástroje lze získat pomocí příkazu:

\begin{lstlisting}[language=bash]
./src/scripts/create_form_manager_user.sh --help
\end{lstlisting}


\section{Správa aplikace}\label{sec:sprava-aplikace}

Aplikace se skládá z několika částí, které je potřeba spravovat.
Zde je seznam částí aplikace a možné způsoby správy:

\begin{description}
    \item[Reverse proxy] Status stránka je dostupná na \lstinline{/nginx_status}.
    Když chceme zjistit zda-li reverse proxy běží, můžeme udělat HTTP GET požadavek na \lstinline{/health}.
    Pokud je návratový kód 200, tak reverse proxy běží.
    \item[Monitoring] Monitoring služeb aplikace je dostupný na \lstinline{/monitoring/}.
    Oficiální dokumentace k webovému rozhraní je dostupná online v anglickém jazyce \href{https://github.com/google/cadvisor/blob/master/docs/web.md}{zde}.
    Monitorovací aplikace poskytuje i API, které je dostupné na \lstinline{/monitoring/api}.
    Oficiální dokumentace k API je dostupná online v anglickém jazyce \href{https://github.com/google/cadvisor/blob/master/docs/api.md}{zde}.
    \item[Form.io] Webové rozhraní a API aplikace Form.io je dostupné na \lstinline{/formio/} po spuštění aplikace.
    Oficiální dokumentace k webovému rozhraní je dostupná online v anglickém jazyce \href{https://help.form.io/}{zde}.
    Oficiální dokumentace k API je dostupná online v anglickém jazyce \href{https://apidocs.form.io/}{zde}.
\end{description}

