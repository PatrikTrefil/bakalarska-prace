\chapter*{Úvod}
\addcontentsline{toc}{chapter}{Úvod}

Tato práce vznikla z motivace zlepšit kvalitu poskytované zdravotní péče v oblasti duševního zdraví za pomoci digitálních technologií.
Léčba duševních potíží se nazývá psychoterapie a trvá obvykle několik měsíců až let.
V průběhu terapie pacient pravidelně dochází na sezení s terapeutem.
Na těchto sezeních terapeut zadává pacientovi různé úkoly, které pacient následně vypracuje mimo sezení.
Úkoly mají často formu dotazníků, mentálních cvičení nebo vzdělávacích aktivit jako je čtení článků.
Tyto úkoly jsou často zadávány papírovou formou a následně ručně vyhodnocovány.
Digitalizací procesu spolupráce lze zajistit efektivnější spolupráci mezi terapeuty a pacienty a automatizovaný sběr dat o pacientech.
Získaná data lze využít pro sledování vývoje stavu pacientů a také pro výzkum.
Cílem je tedy posílení spolupráce mezi terapeuty a jejich pacienty/klienty.

Výzkum z oblasti psychologie může mít mnoho forem.
Jedna z častých forem výzkumu je sběr dat o skupině pomocí dotazníků.
Dotazníky mají často papírovou formu a následně jsou ručně vyhodnocovány.
Digitalizací sběru dat lze zajistit efektivnější sběr dat a automatizované vyhodnocování.
Vzhledem k požadavkům této domény na digitální nástroje jako je nutnost zachování anonymity pacientů/klientů/účastníků studie, není možné použítí obecných nástrojů jako je Google Forms.
Psychoterapie a výzkum v oblasti psychologie jsou oblasti, které mohou získat velký prospěch z digitalizace, ale vyžadují specializované nástroje.

Tato práce byla vytvořena ve spolupráci s \href{https://www.nudz.cz/}{Národním ústavem duševního zdraví (NUDZ)}, který poskytl svou expertízu a zkušenosti v oblasti duševního zdraví.
Tato instituce se zabývá výzkumem neurobiologických a psychosociálních mechanismů spojených se vznikem a průběhem nejzávažnějších duševních poruch~\cite{nudz-profil}.
Ústav současně poskytuje i psychiatrickou péči nemocným a plánuje řešení používat.

Iniciativa ke spolupráci s NUDZ přišla z mé strany a byla přijata s otevřeností.
V rámci spolupráce došlo k plné realizaci procesu vývoje softwaru od sběru požadavků, přes analýzu, design až po implementaci, testování a předání.
Vývoj byl v agilním duchu proložen schůzkami, kde byly prezentovány dosažené výsledky a byla vedena diskuze ohledně dalších kroků.

Nyní popíšeme strukturu práce.
Na začátku provedeme analýzu požadavků Národního ústavu duševního zdraví a popíšeme doménu (kapitola~\ref{ch:analyza-pozadavku}).
Následně prozkoumáme již existující software, který v práci použijeme (kapitola~\ref{ch:analyza-existujicich-reseni-pro-praci-s-formulari}).
Ná základě průzkumu navrhneme řešení (kapitola~\ref{ch:navrh-aplikace}).
Poté popíšeme jakým způsobem jsme postupovali při implementaci a proč jsme tento postup zvolili (kapitola~\ref{ch:vyvojova-dokumentace}).
V další kapitole se zaměříme na testování aplikace (kapitola~\ref{ch:testovani-aplikace}).
Vysvětlíme, jak se aplikace provozuje (kapitola~\ref{ch:administratorska-prirucka}) a jak se aplikace užívá z pohledu terapeta/výzkumníka a pacienta/klienta/účastníka studie (kapitola~\ref{ch:uzivatelska-dokumentace}).
Dále zhodnotíme jednotlivé části procesu vývoje a zhodnotíme celkovou úspěšnost projektu (kapitola~\nameref{ch:zhodnoceni-vyvoje}).
Na závěr shrneme dosažené výsledky a navrhneme možnosti dalšího vývoje (kapitola~\ref{ch:zaver}).