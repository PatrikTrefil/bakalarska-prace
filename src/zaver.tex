\chapter*{Závěr}
\addcontentsline{toc}{chapter}{Závěr}

Na závěr bych se chtěl ohlédnout zpátky a zhodnotit průběh celého vývoje.


\section{Spolupráce s NÚDZ}\label{sec:spoluprace-s-nudz}

Spolupráce s Národním ústavem duševního zdraví nebyla vždy jednoduchá.
Nízká technická znalost byla větší problém než jsem očekával.
V některých chvílích jsem nedokázal klást ty správné otázky a nepodařilo se mi zachytit všechny důležité informace o doméně.
Celá spolupráce byla značně ztížena tím, že se v průběhu vývoje několikrát změnila osoba, která se mnou spolupracovala a komunikovala.
Celkově jsem však spokojený s výsledkem a věřím, že aplikace bude v budoucnu využívána a pomůže zlepšit kvalitu poskytované péče.


\section{Technická stránka}\label{sec:technicka-stranka}

\subsection{Výběr technologií}\label{subsec:vyber-technologii}

Kladně hodnotím všechny volby použitých technologií kromě systému spravující formuláře.
Systém spravující formuláře Form.io na mě na začátku působil velmi profesionálně a vyspěle.
Narazil jsem však na mnoho problémů a nedostatků, které mi značně zkomplikovaly práci.
Software má mnoho dokumentace, ale některé části jsou nesrozumitelné a některé funkce jsou nedostatečně zdokumentované.
Klientské knihovny mají mj.\ špatnou podporu TypeScript, nepodporují všechny funkce a mají nepříjemné rozhraní.
Po několika měsících práce jsem došel k závěru, že některé klientské knihovny budu muset nahradit vlastním řešením.
Dalším problémem byla špatná spolupráce se správci projektu.
V průběhu mé práce jsem narazil na nedostatky software Form.io či dokumentace a tyto nedostatky jsem se vždy nahlásil, či jsem dokonce sám opravil.
Většina z nahlášených chyb dodnes nejsou adresovány a návrhy na mé vylepšení nebyly přijaty.
Kdybych tento projekt začínal znovu, mnohem více bych se snažil zadavateli rozmluvit požadavek na ukládání dat na vlastním serveru.
Tento požadavek je totiž hlavním důvodem proč jsem se nakonec pro Form.io rozhodl.
Myslím si, že existuje mnoho lepších řešení pro správu formulářů, ale jedná se o cloudové služby.
Na druhou stranu, je potřeba říci, že kdybych se rozhodl pro psaní vlastního řešení pro správu formulářů, tak bych pravděpodobně nedosáhl tak dobrého výsledku.
Software Form.io poskytuje opravdu mnoho funkcí a věřím, že to zlepšilo kvalitu výsledné aplikace.

Rád bych ještě popsal svou zkušenost s framework NextJS\.
Framework mi poskytl mnoho skvělých funkcí jako je routing, caching a také sdílení a zanořování layoutů.
Framework má skvělou dokumentaci a velkou aktivní komunitu.
Narazil jsem však i na několik problémů.
První problém byl s hot module replacement.
Tato funkce umožňuje vyměnit, přidat nebo odebrat moduly za běhu aplikace bez nutnosti dalšího načtení stránky.
Build systém reaguje na změny souborů, které nahlašuje souborový systém, a následně aplikuje změny za běhu aplikace.
Pokud však vývojový server pustíme v Docker kontejneru, kde uděláme mount složky s kódem z hostitelského systému Windows, tak se změny v souborech nebudou detekovat.
Tato chyba je známá již od roku 2019 a zatím nebyla opravena (viz \href{https://github.com/microsoft/WSL/issues/4739}{diskuze v Github issue}).
Dalším problémem byla omezenost middleware, který NextJS poskytuje.
Middleware je kód, který se spustí před zpracováním všech požadavků na server.
Tento kód však běží v speciálním optimalizovaném prostředí nazývané edge runtime.
Edge runtime je však velmi omezené a spoustu knihoven na něm nelze použít.
Důsledky toho jsou, že například nelze použít populární knihovny pro logování (\href{https://github.com/pinojs/pino}{Pino}, \href{https://github.com/winstonjs/winston}{Winston}, apod.) ani knihovnu \href{https://github.com/axios/axios}{axios}, která poskytuje alternativu k nativnímu fetch API\.
NextJS v tuto chvíli nedovoluje použít jiné běhové prostředí pro middleware.
Posledním problémem byla implementace content security policy.
Content security policy je bezpečnostní vrstva, která zabraňuje určitým typům útoků jako je například \href{https://developer.mozilla.org/en-US/docs/Glossary/Cross-site_scripting}{cross-site scripting (XSS)}.
Pro použití tohoto bezpečnostního prvku je potřeba nakonfigurovat hlavičky HTTP odpovědí a také modifikovat zdrojový kód stránky.
Podpora této vrstvy je v NextJS aktivně vyvíjena, ale v době psaní této práce nebyla ještě plně funkční.

\subsection{Testování aplikace}\label{subsec:testovani-aplikace}

Serverová část aplikace je testována pomocí unit testů.
Testujeme pouze veřejné rozhraní všech modulů.
Všechny proxy databáze jsou nahrazeny mock objekty a kontrolujeme, zda-li se volají konkrétní metody s očekávanými parametry.
Toto je doporučený přístup object-relational mapping knihovny Prisma, který je popsán v \href{https://www.prisma.io/docs/guides/testing/unit-testing}{dokumentaci}.
Tento přístup se mi vůbec neosvědčil.
Kontrola volání konkrétních metod vytváří obrovskou závislost na vnitřní implementaci testovaných metod.
Testy jsou velmi těžko udržovatelné a navíc poměrně dlouhé a složité.
Kdybych měl začínat znovu, tak bych se snažil primárně testovat privátní metody, které obsahují veškerou logiku mimo databázové operace.
Pro testování metod obsahující databázové operace bych zvážil použití in-memory databáze.

\subsection{Open-source vývoj}\label{subsec:open-source-vyvoj}

Při vývoji jsem narazil na mnoho chyb a nedostatků v použitých nástrojích a knihovnách.
Tyto chyby jsem hlásil správcům jednotlivých projektů a některé jsem dokonce sám opravil.
V případě některých projektů byla komunikace se správci velmi dobrá a chyby byly rychle opraveny.
V některých případě jsem však narazil na rozdílnosti v názorech na správné řešení problémů nebo jsem se nedočkal žádné reakce.
Zde je výčet issues a pull requestů, které jsem vytvořil:

\begin{itemize}
    \item
    \href{https://github.com/formio/react/issues/522}{Chyba v internacionalizaci komponenty poskytované knihovnou}
    \item
    \href{https://github.com/formio/react/pull/538}{Oprava chyby v internacionalizaci komponenty poskytované knihovnou}
    \item
    \href{https://github.com/formio/formio-app-formio/issues/35}{Bezpečnostní chyba v Form.io klientské aplikace}
    \item
    \href{https://github.com/formio/formio/issues/1555}{Chybějící dokumentace API endpointu}
    \item
    \href{https://github.com/formio/formio/issues/1485}{Chyba ve verzi databáze}
    \item
    \href{https://github.com/formio/formio-app-formio/issues/34}{Návrh na zlepšení organizace repozitáře}
    \item
    \href{https://github.com/formio/react/issues/523}{Použití zastaralého API}
    \item
    \href{https://github.com/tgreyuk/typedoc-plugin-markdown/issues/429}{Chyba v pluginu do Docusaurus}
    \item
    \href{https://github.com/tgreyuk/typedoc-plugin-markdown/issues/440}{Návrh na vylepšení pluginu do Docusaurus}
    \item
    \href{https://github.com/react-bootstrap/react-bootstrap/issues/6671}{Nefunkční příklad v dokumentaci}
    \item
    \href{https://github.com/gajus/eslint-plugin-jsdoc/issues/1138}{Návrh na vylepšení chování statické analýzy kódu}
    \item
    \href{https://github.com/MrFlynn/upload-to-netlify-action/issues/17}{Chyba v dokumentaci GitHub Action pro upload na Netlify}
\end{itemize}


\section{Shrnutí}\label{sec:shrnuti}

V této práci jsme se zabývali vývojem webové aplikace pro monitorování mentálního zdraví.
Na začátku jsme detailně analyzovali požadavky zadavatele a vytvořili návrh aplikace.
Poté jsme rozebrali proces vývoje, popsali technické detaily a sepsali uživatelskou dokumentaci.
Na závěr jsme zhodnotili průběh celého vývoje a výslednou aplikaci.