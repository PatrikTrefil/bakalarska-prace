\chapter{Vývojová dokumentace}\label{ch:vyvojova-dokumentace}

Tato kapitola chronologicky popisuje vývoj aplikace.
Navrhneme uživatelské rozhraní formou wireframů, vybereme technologie, které budeme používat a také si rozmyslíme architekturu.
Na závěr popíšeme deployment aplikace.


\section{Návrh řešení}\label{sec:navrh-reseni}

Jelikož chceme, aby byla aplikace dostupná jak na počítačích, tak na mobilních zařízeních, nabízí se dvě možnosti.
První možností je vyvinout mobilní aplikaci a desktopovou aplikaci zvlášť.
Druhou možností je vyvinout webovou aplikaci s responzivním rozhraním, která bude dostupná na všech zařízeních.
Jelikož nepotřebujeme žádné nativní funkce zařízení, tak je výhodnější vyvinout pouze jednu webovou aplikaci.
Webová aplikace má také výhodu v tom, že ji není potřeba instalovat a aktualizovat.
Nezapomeňme zmínit, že zadavatel má již zkušenosti s provozem webových aplikací a má již existující infrastrukturu.

Budeme se snažit použít co nejvíce již existujícího software, abychom se vyhnuli duplicitě práce a zároveň zjednodušili údržbu aplikace.
Naším cílem bude vybrat software, který nám dá dostatek flexibility a zároveň bude dostatečně stabilní a udržitelný.
Nemáme rozpočet na použití placeného software a proto budeme vybírat pouze z open-source řešení.
Systém chceme navrhnout tak, aby bylo možné v budoucnu vyměnit některé komponenty za komplexnější placené řešení.

Abychom zajistili spokojenost zadavatele, budeme se snažit pravidelně prezentovat výsledky vývoje a získávat zpětnou vazbu.
Na začátku vývoje budeme směřovat k vytvoření minimálního životaschopného produktu, který bude mít základní funkce.


\section{Návrh uživatelského rozhraní}\label{sec:navrh-uzivatelskeho-rozhrani}

Jak již víme z analýzy požadavků budeme potřebovat navrhnout rozhraní pro plnitele úkolů a také pro terapeuty.
Předpokládáme, že uživatelé mají nízké technické dovednosti a proto se budeme snažit navrhnout co nejjednodušší rozhraní.
Abychom předešli vývoji rozhraní, které by zadavatel nepovažoval za vhodné, tak vytvoříme wireframy, které necháme zadavatelem schválit.
Výsledné wireframy naleznete v přílohách práce\ \ref{sec:wireframy-uzivatelskeho-rozhrani}.


\section{Výběr technologií}\label{sec:vyber-technologii}

V této sekci se budeme zabývat výběrem technologií, které budeme používat při vývoji aplikace.
Chceme volit populární nástroje, jelikož programátoři, kteří budou aplikaci v budoucnu rozšiřovat, je pravděpodobně budou znát.
Zároveň také chceme co nejméně komplikovat workflow a deployment, abychom zbytečně nezvyšovali nároky na programátory, kteří budou aplikaci v budoucnu rozšiřovat.
Jelikož nemáme rozpočet na placené aplikace, tak musíme vybírat primárně z open-source řešení a nebo vyjednat smlouvu s komerčními firmami.

\subsection{Software pro práci s formuláři}\label{subsec:software-pro-praci-s-formulari}

Software pro práci s formuláři je klíčovou součástí aplikace.
Musíme zvážit v jakém formátu definice formulářů ukládat, jaký software použít pro tvorbu formulářů, vykreslení formulářů a sběr odpovědí.

Existuje mnoho knihoven pro vykreslování formulářů na základě schématu.
Existuje však velmi málo knihoven pro tvorbu formulářů.
Knihovny pro tvorbu formulářů jsou náročné na vývoj a proto jsou velmi často součástí pouze placených řešení.

Jedna z takových knihoven je vyvíjena firmou \href{https://www.vazco.eu/}{Vazco}.
Knihovna je však součástí placeného produktu.
Firmu jsem kontaktoval a dostal jsem potvrzení, že můžeme jejich knihovnu pro tento projekt použít.
Hned po obdržení potvrzení však firma přestala reagovat na mé e-maily.
Navzdory několika dalším pokusům o kontaktování firmy dopadlo toto jednání neúspěšně.

Další alternativou je knihovna \href{https://formilyjs.org/}{Formily}.
Formily je velký open-source projekt, který dosáhl poměrně velké popularity.
Za projektem navíc stojí velká komerční firma Alibaba.
Projekt je aktivně vyvíjen, ale bohužel v době volby technologie nebyla dostupná dokumentace v anglickém jazyce.
Z důvodu chybějící dokumentace jsem tuto knihovnu musel zavrhnout.

Posledním velkým projektem z oblasti knihoven pro tvorbu formulářů, který jsem našel je \href{https://form.io/}{Form.io}.
Tento projekt byl již zmíněn v sekci o již existujících řešení\ \ref{ch:analyza-existujicich-reseni-pro-praci-s-formulari}.
Tentokrát se nebudeme dívat na placené služby, ale na open-source jádro projektu, které má licenci Open Software License 3.0, což dovoluje komerční použití.
Zásadní výhodou tohoto projektu je, že řeší všechny problémy týkající se správy formulářů - formát definic formulářů, tvorbu formulářů, vykreslení formulářů a sběr odpovědí.
Produkt má spoustu dokumentace, kvalitní API a je stále aktivně vyvíjen.
Řešení navíc obsahuje základní systém pro správu uživatelů.
Projekt je zastřešen komerční firmou, což zvyšuje jeho dlouhodobou udržitelnost.
Nevýhodou je, že použití vlastního autentifikačního poskytovatele je dostupné pouze v placených verzích.
Software Form.io je využíván pro provoz pohraničních sil Britské vlády (viz \href{https://www.youtube.com/watch?v=nuf46N5vU34}{video z konference CamundaCon}).
Dle oficiálních stránek Form.io je také využíván organizacemi jako Visa či ICANN\@.

Po zvážení těchto možností jsem se rozhodl použít jádro produktu Form.io.
Některé placené funkce by použití ulehčily, ale nejsou pro tento projekt nezbytně nutné.
Zároveň se nabízí jednoduchý přechod na placenou verzi po nárazu na limity bezplatné verze.

\subsection{Frontend aplikace}\label{subsec:frontend-aplikace}

\begin{description}
    \item[Bootstrap]
    Jednoduchý způsob jak pracovat s CSS\@.
    Knihovna je velmi populární, takže je pravděpodobné, že ji ostatní programátoři znají.
    Alternativy jako Tailwind CSS nabízejí větší míru flexibility, ale neposkytují žádné hotové komponenty.
    Tento projekt nemá za cíl vytvářet vlastní designovou identitu, proto je výhodnější použít hotové komponenty, které Bootstrap nabízí.
    \item[JavaScript/TypeScript]
    Výchozí volba pro programování webových stránek.
    TypeScript poskytne statickou typovou kontrolu, což pomůže předejít mnoha chybám.
    Přestože TypeScript zkomplikuje a zpomalí vývoj, tak se, dle mého názoru, z dlouhodobého hlediska vyplatí jej používat.
    \item[React]
    React je dle průzkumu \href{https://2022.stateofjs.com/en-US/libraries/front-end-frameworks/}{State of JavaScript} z roku 2022 nejpoužívanější front-end framework.
    React je vyspělý framework, který má velkou komunitu a mnoho existujících knihoven.
    \item[NextJS]
    Místo volby jednotlivých balíků pro základy řešení routing, middleware, sdílených layoutů apod.\ zvolme populární framework, který všechny tyto funkce poskytuje.
    Jedná se o full-stack framework, takže jej využijme i na serverovou část.
    Tím se vyhneme nutnosti zakládat další projekt.
    \item[react-bootstrap]
    Tato knihovna značně zjednodušuje použití knihovny Bootstrap v React aplikacích a navíc zajistí alespoň základní webovou přístupnost.
    \item[react-i18next]
    Aplikace bude podporovat více jazyků a proto je vhodné použít knihovnu, která nám usnadní práci s překlady.
    Oproti alternativám jako \href{https://github.com/airbnb/polyglot.js}{Polyglot.js} nebo \href{https://github.com/formatjs/formatjs}{Format.js} nabízí knihovna react-i18next více funkcí.
    Knihovna react-i18next je s 2.9 milióny stažení za týden nejpopulárnější ze zvažovaných knihoven dle počtu stažení z npm registry.
    \item[React formio]
    Knihovna pro vykreslování formulářů na základě schématu.
    Tato oficiální knihovna je součástí projektu Form.io.
    \item[NextAuth.js]
    Knihovna pro autentifikaci uživatelů v prohlížeči i na serveru.
    Ukázalo se, že Formio React není vhodný pro autentifikaci uživatele.
    První důvod je, že vyžaduje, aby si vývojář psal vlastní logiku pro ochranu stránek, přesměrování apod.
    To zbytečně vytvářelo prostor pro chyby.
    Druhý důvod je ten, že inicializace autentifikace se dělala pouze na klientovi, jelikož knihovna nepodporuje server-side rendering, což mělo negativní vliv na výkon aplikace.
    Třetí důvod je špatná dokumentace knihovny.
    Jako alternativy jsem zvažoval Passport.js a NextAuth.
    Passport.js je více nízko-úrovňový a opět vyžaduje, aby si vývojář psal vlastní logiku.
    Oproti Formio React má však pěknou dokumentaci.
    Poslední výhodou použití jiné knihovny než Formio React je to, že máme v kódu vrstvu abstrakce a tedy kód webové aplikace není závislý na konkrétním autetifikačním systému.
    \item[Zod]
    Pro validaci dat na serveru a validaci formulářů na klientovi využijeme validační knihovnu.
    Knihovnu Zod jsem zvolil jelikož má rozhraní, se kterým se dobře pracuje.
    \item[React hook form]
    Aplikace bude obsahovat formuláře pro přihlášení, správu uživatelů, tvorbu úkolů a mnoho dalších.
    Pro lepší práci s formuláři využijeme knihovnu.
    Díky této knihovně budeme moci snadno validovat formuláře, získávat hodnoty formulářů a zobrazovat chybové hlášky.
    Lze zvážit také knihovnu \href{https://formik.org}{Formik}.
    Recenze na internetu byly stejně dobré jako pro React hook form, ale integrace s validační knihovnou Zod existuje pouze jako \href{https://github.com/robertLichtnow/zod-formik-adapter}{komunitní balíček}.
    React hook form má oficiální integraci s validační knihovnou zod pomocí \href{https://github.com/react-hook-form/resolvers}{balíčku resolvers}.
    \item[Tanstack query]
    Aplikace bude obsahovat mnoho dat, která budou načítána ze serveru.
    Pro zvýšení kvality kódu a zlepšení výkonu aplikace využijeme knihovnu.
    Populární řešení jsou Tanstack query a SWR\@.
    Tanstack query nabízí více funkcí a mnoho z nich nám značně ulehčí práci.
    Tanstack query například nabízí podporu pro stránkování.
    \item[Tanstack table]
    Aplikace bude obsahovat velké množství tabulek.
    Vykreslování velkých tabulek efektivně je náročný úkol.
    Vzhledem k tomu, že používáme knihovnu Tanstack query, tak se nabízí použít i knihovnu Tanstack table.
    Výhodou této knihovny je, že je velmi flexibilní a umožňuje vytvářet vlastní komponenty pro vykreslování tabulek.
    Jedná se o tzv. \textit{headless UI} knihovnu.
    \item[Recharts]
    Aplikace bude potřebovat vizualizovat sesbíraná data a proto potřebujeme knihovnu na vykreslování grafů.
    Pro tyto účely byly zváženy knihovny \href{https://github.com/airbnb/visx}{visx} od AirBnB, \href{https://github.com/apache/echarts} od Apache a \href{https://github.com/recharts/recharts}{Recharts}.
    Knihovna visx má velmi komplikovanou dokumentaci i příklady.
    Knihovna má evidentně strmou výukovou křivku a pro naše účely je zbytečně komplexní.
    Knihovna echarts je velmi populární, ale nemá oficiální podporu pro React.
    Knihovna Recharts má kvalitní dokumentaci obsahující jak jednoduché tak složitější příklady a navíc má přímou podporu pro React.
\end{description}

\subsection{Middleware}\label{subsec:middleware}

Pro komunikaci mezi klientem a serverem využijeme middleware.
To nám automaticky zajistí typovou bezpečnost, serializaci dat a zlepší vývojářskou zkušenost.
Abychom neztratili interoperabilitu serveru s ostatními aplikacemi, použijeme knihovnu, která umožní vytvořit i REST API\@.
Mezi populární volby řešení patří knihovny \href{https://grpc.io/}{gRPC} od firmy Google, \href{https://www.npmjs.com/package/json-rpc-2.0}{json-rpc-2.0} implementující \href{https://www.jsonrpc.org/specification}{standard JSON-RPC 2.0}  a \href{https://trpc.io/}{tRPC}.
Knihovna gRPC však nefunguje v prohlížeči (viz \href{https://grpc.io/blog/state-of-grpc-web/}{článek}).
Knihovna tRPC má skvělou podporu pro TypeScript a je kompatibilní s knihovnou NextAuth.js.
Knihovna tRPC je oproti knihovně json-rpc-2.0 značně populárnější, nabízí více funkcí a má aktivnější vývoj.

\subsection{Backend aplikace}\label{subsec:backend-aplikace}

\begin{description}
    \item[nginx]
    Veškeré požadavky na server budou směrovány přes reverse proxy.
    To nám umožní konfigurovat routing, SSL certifikáty apod.
    Mezi další zvážené možnosti patří \href{https://httpd.apache.org/}{Apache HTTP Server}.
    Po prostudování dokumentace obou nástrojů jsem dospěl k názoru, že nginx má lepší dokumentaci a čitelnější formát konfigurace.
    Nginx má navíc největší podíl z počítačů na internetu dle \href{https://news.netcraft.com/archives/category/web-server-survey/}{průzkumu z března 2023}.
    \item[NextJS]
    Důvody popsány v sekci~\ref{subsec:frontend-aplikace}.
    \item[Form.io]
    Důvody popsány v sekci~\ref{subsec:software-pro-praci-s-formulari}.
    \item[MongoDB]
    Form.io server podporuje pouze MongoDB pro ukládání dat\@.
    \item[PostgreSQL]
    Potřebujeme databázi pro ukládání dat o úkolech a také nedokončené odpovědi na dotazníky.
    Chceme bezplatné řešení, které je dostatečně stabilní a jeho licence umožňuje komerční použití.
    Jelikož naše data lze dobře modelovat pomocí relací jak uvidíme v sekci x. % TODO: add reference
    tak použijeme relační databází.
    Výběr konkrétního řešení není příliš důležitý z následujících důvodů.
    Naše nároky na databázový systém jsou poměrně nízké a navíc budeme používat abstrakci nad databází, která nám umožní databázový systém kdykoliv vyměnit.
    PostgreSQL patří mezi nejlepší řešení splňující všechny naše požadavky.
    \item[Prisma]
    Abychom zajistili dobrou modifikovatelnost celého systému použijeme techniku object-relational mapping.
    Mezi populární volby patří knihovny \href{https://github.com/drizzle-team/drizzle-orm}{Drizzle ORM}, \href{https://github.com/typeorm/typeorm}{TypeORM} a \href{https://github.com/prisma/prisma}{Prisma}.
    Knihovna Prisma je nejvyspělejší a nejpopulárnější z těchto knihoven co se týče počtu stažení za týden z npm registry.
    Knihovna Prisma je navíc vyvíjena komerční firmou, což zvyšuje její dlouhodobou udržitelnost.
\end{description}

\subsection{Nástroje}\label{subsec:nastroje}

\begin{description}
    \item[Prettier]
    Formátování kódu.
    \item[ESLint]
    Statická analýza kódu.
    \item[Github Workflows]
    Continuous integration.
    \item[TypeDoc]
    Implementuje generování dokumentace z dokumentačních komentářů dle standardu \href{https://tsdoc.org/}{TSDoc}.
    Alternativou je \href{https://github.com/dotnet/docfx}{DocFX}, ale to je nástroj napsaný v C\#.
    Tento nástroj nemá npm balíček, což komplikuje integraci do našeho projektu.
    Jelikož TypeDoc implementuje stejný standard a jeho použití je v našem případě jednodušší, zvolil jsem jej pro tento projekt.
    \item[Vitest]
    Původně jsem používal pro testování knihovnu \href{https://jestjs.io/}{Jest}, ale narazil jsem na problémy s konfigurací.
    Složité chybové hlášky a nedostatečná dokumentace mě přiměly zkusit jinou knihovnu.
    Zvolil jsem knihovnu Vitest, se kterou jsem měl vělmi dobrou zkušenost.
    \item[Docusaurus]
    Původně jsem používal \href{https://docs.github.com/en/communities/documenting-your-project-with-wikis/about-wikis}{Github Wiki}, ale časem se ukázalo, že je pro tento projekt nevhodná.
    Github Wiki chybí podpora pro diagramy jako kód a také nelze zahrnout automaticky generovanou dokumentaci.
    Proto jsem zvažoval další alternativy jako \href{https://www.mkdocs.org/}{MkDocs}, \href{https://www.gitbook.com/}{GitBook} a \href{https://docusaurus.io/}{Docusaurus}.
    Docusaurus působil moderně, jednoduše a všechny jeho funkce jsou zdarma.
    Docusaurus má mnoho rozšíření jako například podporu pro diagramy jako kód.
    Docusaurus má oficiální plugin pro \href{https://mermaid.js.org/}{MermaidJS}, což je nástroj pro tvorbu diagramů.
    Ukázalo se, že MermaidJS není vhodný pro náš projekt, protože nemá dobrou podporu \href{https://c4model.com/}{C4 diagramu}, které používáme pro dokumentaci architektury.
    Díky velkému ekosystému nástroje Docusaurus však existuje i komunitní plugin pro podporu PlantUML diagramů, který má skvělou podporu C4 diagramů.
\end{description}