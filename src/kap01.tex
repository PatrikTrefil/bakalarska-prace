\chapter{Analýza požadavků}\label{ch:analyza-pozadavku}

Pro analýzu požadavků proběhlo několik online schůzek s pracovníky Národního ústavu duševního zdraví.
Na první schůzce se probíraly možnosti využití digitálních technologií v oblasti duševního zdraví.
Na této schůzce jsme se shodli na tvorbě platformy pro digitalizaci práce s dotazníky.
Výsledkem několika dalších schůzek byla neformální specifikace požadavků.
Doména se ukázala býti poměrně složitá a bylo důležité detailně zdokumentovat její fungování.
Dalším krokem bylo tedy vytvoření doménového modelu, který dostal formu UML diagramu a textového souboru definic entit.
Tato dokumentace nám poskytla společný jazyk pro další komunikaci.
V průběhu tvorby modelu došlo k ujasnění některých požadavků a fungování domény.
Časem byla neformální specifikace převedena do formy user stories.
Tato konverze byla provedena zejména pro lepší organizaci vývoje softwaru.
Dále jsme identifikovali stakeholdery a sepsali jejich zájmy.
Všechny tyto artefakty se nachází v následujících podkapitolách.

\section{Specifikace}\label{sec:specifikace}

Tato sekce obsahuje neformální specifikaci požadavků na aplikaci.
Specifikace vznikla v rané fázi vývoje a některé požadavky byly později změněny.
Změny jsou popsány v sekci~\ref{subsec:zmeny-specifikace}.

\subsection{Skupiny uživatelů}\label{subsec:skupiny-uzivatelu}

\begin{itemize}
\item
Plnitel = pacient/klient

\item
  Zaměstnanci NUDZ

  \begin{itemize}
  \item
    Správce dotazníků = terapeut/výzkumník
  \item
    Zadavatel dotazníků = terapeut/výzkumník
  \end{itemize}
\item
  Technická podpora
\end{itemize}


\subsection{Funkční požadavky}\label{subsec:funkcni-pozadavky}
\begin{itemize}
\item
Zadavatel/správce dotazníků může zadávat dotazníky a úkoly konkrétnímu plniteli, nastavit frekvenci opakování a nastavit start/deadline.

  \begin{itemize}
  \item
    Start je čas, kdy lze dotazník nejdříve vyplnit.
  \item
    Deadline je čas, kdy lze dotazník nejpozději vyplnit.
  \item
    Dotazník se skládá z otázek, které mohou přijímat odpovědi těchto typů:

    \begin{itemize}
    \item
      text
    \item
      více možností (možno zvolit právě jeden)
    \item
      více možností (možno zvolit libovolný počet)
    \end{itemize}
  \item
  U každé otázky je možno nastavit podmíněné zobrazení.
  Jinými slovy, otázka se plniteli zobrazí, pokud je splněna podmínka definovaná správcem dotazníku při vytvoření.
  \end{itemize}
\item
Správce dotazníků je schopen vytvářet/mazat dotazníky a způsoby vyhodnocení
\item
Správce dotazníků může v dotazníku upravit obsah otázek a možných odpovědí v případě, že se jedná o volbu z možností.
\textit{Nelze} přidat/odebrat otázky, přidat/odebrat možné odpovědi, ani upravit způsob výpočtu skóre.

  \begin{itemize}
  \item
    Při tvorbě dotazníku je možno nastavit automaticky počítané metriky pomocí vzorce (např.\ \texttt{otázka1\ +\ otázka2\ -\ otázka4}, kde \texttt{otázka1}, \texttt{otázka2} a \texttt{otázka4} jsou proměnné reprezentující výsledky příslušných otázek).
  \end{itemize}
\item
Plnitel může v aplikaci vypracovat dotazníky a úkoly, které mu byly přiděleny správcem/zadavatelem dotazníků.

  \begin{itemize}
  \item
  Plnitel je schopen vyplnit část dotazníku, uložit si dosud zodpovězené otázky a v budoucnu vyplňování dotazníku dokončit.
  \end{itemize}
\item
Plnitel může smazat svá data.
\item
Správce/zadavatel dotazníků je schopen vidět výsledky dotazníků a úkolů všech plnitelů.
\item
  Správce/zadavatel může vybrat data, která budou vizualizována na grafech.
\item
Správce dotazníků/člen technické podpory je schopen vytvářet/mazat účty plnitelů, účty pro ostatní správce a účty pro zadavatele dotazníků.
\item
  Správce dotazníků/člen technické podpory je schopen měnit přístupová práva všech ostatních účtů.
\item
Plintel je schopen měnit heslo svého účtu a je schopen svůj účet smazat.
\item
Výsledky dotazníků plnitelů může správce/zadavatel dotazníků z aplikace exportovat do formátu CSV\@.
\item
  Aplikace bude pouze v Českém jazyce, ale bude připravena na internacionalizaci.
\item
Plnitel v systému vystupuje pod ID, které je náhodně vygenerováno.
\end{itemize}

\subsubsection{Viditelnost vyhodnocení dotazníků}\label{subsubsec:viditelnost-vyhodnoceni-dotazniku}

\begin{itemize}
\item
Plnitel je schopen vidět vyhodnocení svých dotazníků, u kterých to správce povolil.
\item
Správce dotazníků může dovolit plniteli vidět výsledky konkrétního dotazníku - číselná hodnota/graf/text.
\item
  Viditelnost výsledků lze nastavit následujícími způsoby:

  \begin{itemize}
  \item
  plnitel vidí výsledky
  \item
  plnitel nevidí výsledky
  \item
  plnitel vidí výsledky pouze výsledek dotazníku splňuje podmínku (např.\ výsledné skóre je větší než 10)
  \end{itemize}
\end{itemize}


\subsubsection{Volitelné}

\begin{itemize}
\item
Sledování délky času stráveného nad jednotlivými otázkami, času, kdy plnitel dotazník vyplnil, počtu změn odpovědí.
\item
  E-mail upozornění na nové úkoly, na nesplněné úkoly.
\item
  Push notifikace v prohlížeči na nové úkoly, na nesplněné úkoly.
\item
Chat mezi terapeutem a plnitelem.
\item
  Rozšíření již vystavěné infrastruktury pro zadávání dotazníků pro zadávání cvičení (např.\ přečíst článek, zhlédnout video nebo vypracovat interaktivní cvičení).
\end{itemize}


\subsection{Nefunkční požadavky}\label{subsec:nefunkcni-pozadavky}

\begin{itemize}
\item
Uživatelské rozhraní by mělo být vhodné pro uživatele s nízkými technickými znalostmi.
\item
  K programu by měla být dodána uživatelská a technická dokumentace.
\end{itemize}


\subsection{Deployment}\label{subsec:deployment}

\begin{itemize}
\item
  Aplikace bude spouštěna v kontejneru.
\item
  Parametry serveru: (lze navýšit v případě potřeby)

  \begin{itemize}
  \item
    CPU: 2 jádra
  \item
    RAM: 4 GB
  \item
    HDD: volných 29 GB
  \end{itemize}
\end{itemize}


\subsection{Monitoring}\label{subsec:monitoring}

\begin{itemize}
\item
  Bude k dispozici rozhraní pro monitorování aplikace.
\end{itemize}

\subsection{Změny specifikace}\label{subsec:zmeny-specifikace}

V průběhu vývoj se ukázalo, že některé požadavky popsané v sekci~\ref{subsubsec:viditelnost-vyhodnoceni-dotazniku} lze zjednodušit.
Původní specifikace pracovala s možností nastavení viditelnosti vyhodnocení dotazníků.
Možnost s nastavením viditelnosti na základě splnění podmínky byla zavrhnuta z následujícího důvodu.
Pokud by plnitel neviděl vyhodnocení dotazníku, tak by si mohl odvodit, že výsledek dotazníku je nepříznivý.
Toto zjištění by mohlo mít negativní vliv na jeho psychiku.
Možnost nastavit vyhodnocení dotazníku jako vždy viditelné pro plnitele bylo zavrhnuto ze stejného důvodu.
Vyhodnocení složitějších dotazníků navíc může správně interpretovat pouze odborník.

% TODO: uživatel nemůže mazat svá data (vědecké účely viz sekce GDPR)

\section{Doménový model}\label{sec:domenovy-model}

Nyní popíšeme doménu formou diagramu v notaci \href{https://www.omg.org/spec/UML/2.5.1/PDF}{UML} (Obr.~\ref{fig:domain-model}).

\begin{figure}[H]
    \includegraphics[width=0.9\textwidth]{diagrams/domainModel}
    \caption{Doménový model}\label{fig:domain-model}
\end{figure}

\begin{tcolorbox}
Dědičnost v UML vyjadřuje jiný vztah než v programování.
Třída v UML je množina a odvozená třída je její podmnožina.
Vztahy mezi odvozenými třídami, tedy podmnožinami, mohou být 4 typů podle toho zda-li se podmnožiny překrývají a zda-li je jejich sjednocení množina definovaná třídou, od které odvozujeme.
Tento vztah je vždy popsán v poznámce k třídě, od které odvozujeme (\href{https://www.omg.org/spec/UML/2.5.1/PDF}{zdroj}).
\end{tcolorbox}

\subsection{Definice entit}\label{subsec:definiceentit}

\begin{description}
\item[Osoba]
  Lidská bytost, která má vztah k systému.

\item[Zaměstnanec]
  Osoba, která je zaměstnána v NUDZ\@.

\item[SprávceÚkolů]
  Zaměstnanec, který je zodpovědný za správu úkolů v systému.

\item[ZadavatelÚkolů]
  Zaměstnanec, který zadává pro plnitelé úkoly v systému.

\item[Plnitel]
  Osoba, která vypracovává úkoly v systému.

\item[Klient]
  Plnitel, který je klientem NUDZ (platí za poskytovanou péči).

\item[Pacient]
  Plnitel, který je pacientem NUDZ (poskytovaná péče je hrazena pojišťovnou).

\item[DefiniceÚkol]
  Obecný potenciálně znovupoužitelný popis činnosti plnitele.

\item[DefiniceFormuláře]
  Definice úkolu, která obsahuje formulář, který je určen k vyplnění plnitelem.

\item[ZadáníÚkolu]
  Zadání úkolu pro konkrétního plnitele na základě definice úkolu.

\item[VypracováníÚkolu]
  Vypracování úkolu plnitelem na základě zadání úkolu.

\item[VypracováníFormuláře]
  Vypracování úkolu ve tvaru vyplnění formuláře.

\item[AnalýzaVypracováníFormuláře]
Odvozená data z vypracování formuláře (např.\ výsledné skóre).

\item[NedokončenéVyplněníFormuláře]
Částečné vyplnění formuláře.
(Umožňuje plniteli přerušit vyplňování formuláře a pokračovat později.)

\end{description}

\subsection{Omezení}\label{subsec:omezeni}

Zde jsou vypsána omezení, která nejsou vyjádřena v diagramu.
Každé omezení je napsáno jak v neformální textové podobě, tak v \href{https://www.omg.org/spec/OCL/2.4/About-OCL}{Object constraint language}.

\subsubsection{NedokončenéVyplněníFormuláře}

Nedokončené vyplnění formuláře pro definici formuláře, zadání úkolu a plnitele může existovat pouze pokud je zadání úkolu pro plnitele a zadání úkolu zadává stejný formulář, který je částečně vyplněn nedokončeným vyplněním formuláře.

\begin{verbatim}
context plnitel: Plnitel inv
  plnitel->vlastni->forAll(
      nedokonceneVyplneniFormulare |
          nedokonceneVyplneniFormulare
            ->castecneVypracovava
            ->zadanPro = self
          and
          nedokonceneVyplneniFormulare
            ->castecneVypracovava
            ->definovano = nedokonceneVyplneniFormulare->vyplnuje
          )
\end{verbatim}

\subsubsection{ZadáníÚkolu}

Definice úkolu musí logicky odpovídat vypracování úkolu.
Např.\ nemůžeme považovat přečtení článku jako vypracování úkolu, který je definován jako vypracování formuláře.

\begin{verbatim}
context z: ZadáníÚkolu inv
    if z.vypracování.oclIsKindOf(VypracováníFormuláře) then
        z.definováno.oclIsKidnOf(DefiniceFormuláře)
    endif
\end{verbatim}

\subsubsection{VypracováníÚkolu}

ID musí být unikátní.

\begin{verbatim}
context v1, v2: VypracováníÚkolu inv v1.ID = v2.ID implies v1 = v2
\end{verbatim}

\subsubsection{DefiniceÚkolu}

ID musí být unikátní.

\begin{verbatim}
context d1, d2: DefiniceÚkolu inv d1.ID = d2.ID implies d1 = d2
\end{verbatim}

\subsubsection{Osoba}

ID musí být unikátní.

\begin{verbatim}
context o1, o2: Osoba inv o1.ID = o2.ID implies o1 = o2
\end{verbatim}

\section{User stories}\label{sec:user-stories}

Zde je seznam funkčních požadavků ve standardizované formě.
Požadavky mají formu \uv{Jako \textless{}role\textgreater{} chci \textless{}funkčnost\textgreater{}, abych/protože/aby \textless{}cíl\textgreater{}}.
Jedná se o přepis neformální specifikace ze sekce~\ref{sec:specifikace}.
Tato sekce pracuje již s změnami specifikace, které byly popsány v sekci~\ref{subsec:zmeny-specifikace}.

\begin{itemize}
  \item
  Jako zaměstnanec chci mít možnost založit účet pro plnitele, abych mohl využít systém v rámci terapie/výzkumu.
  \item
  Jako plnitel chci, aby moje data nemohli číst ostatní plnitelé, protože jsou soukromá.
  \item
  Jako terapeut/výzkumník chci aby byl plnitel schopen plnit pouze úkoly, které mu byly zadány, aby se zachovala integrita sbíraných dat.
  \item
  Jako plnitel chci být schopen zobrazit svá data, abych věděl co je o mě v systému evidováno (např.\ vyplněné dotazníky).
  \item
  Jako správce dotazníků chci mít možnost zadávat dotazníky a úkoly konkrétnímu plniteli, nastavit frekvenci opakování a nastavit start/deadline, abych mohl správně koordinovat procesy a aktivity v systému.
  \item
  Jako správce dotazníků chci mít možnost vytvářet otázky v dotaznících, které mohou přijímat odpovědi typu text, více možností (možno zvolit právě jeden) a více možností (možno zvolit libovolný počet), abych mohl vytvářet různorodé a komplexní dotazníky.
  \item
  Jako správce dotazníků chci mít možnost nastavit podmíněné zobrazení otázky, takže otázka se plniteli zobrazí pokud je splněna podmínka definovaná mnou při vytvoření, což umožní flexibilní a přizpůsobené průzkumy.
  \item
  Jako správce dotazníků chci mít možnost vytvářet, mazat dotazníky a způsoby vyhodnocení, abych mohl udržovat aktuální a relevantní soubor dotazníků.
  \item
  Jako správce dotazníků chci mít možnost nastavit automaticky počítané metriky pomocí vzorce (např.\ \texttt{otázka1 + otázka2 - otázka4}), abych mohl efektivně vyhodnocovat odpovědi.
  \item
  Jako plnitel chci mít možnost vypracovat dotazníky a úkoly, které mi byly přiděleny správcem/zadavatelem dotazníků, abych mohl aktivně participovat v systému.
  \item
  Jako plnitel chci mít možnost vyplnit část dotazníku, uložit si dosud zodpovězené otázky a v budoucnu vyplňování dotazníku dokončit, aby bylo vyplňování dotazníků flexibilní a pohodlné.
  \item
  Jako plnitel chci mít možnost smazat svá data, abych měl kontrolu nad svými daty.
  \item
  Jako správce/zadavatel dotazníků chci mít možnost vidět výsledky dotazníků a úkolů všech plnitelů a vybrat data, která budou vizualizována na grafech, abych mohl monitorovat a analyzovat výsledky.
  \item
  Jako správce dotazníků/člen technické podpory chci mít možnost vytvářet/mazat účty plnitelů, účty pro ostatní správce a účty pro zadavatele dotazníků, abych mohl efektivně spravovat uživatelské účty.
  \item
  Jako správce/zadavatel dotazníků chci mít možnost exportovat výsledky dotazníků plnitelů z aplikace do formátu CSV, abych mohl dělat pokročilé analýzy dat.
  \item
  Jako plnitel chci, aby aplikace byla v Českém jazyce, ale byla připravena na internacionalizaci, abych mohl používat aplikaci v jazyce, kterému rozumím.
  \item
  Jako plnitel chci, aby uživatelské rozhraní bylo vhodné pro uživatele s malými technickými znalostmi, abych mohl snadno používat aplikaci bez ohledu na své technické dovednosti.
  \item
  Jako plnitel chci, aby k programu byla dodána uživatelská a technická dokumentace, abych mohl lépe porozumět, jak aplikace funguje a jak ji používat.
  \item
  Jako technický pracovník chci, aby aplikace byla spouštěna v kontejneru, abych mohl snadno spravovat a nasazovat aplikaci.
  \item
  Jako technický pracovník chci mít k dispozici rozhraní pro monitorování aplikace, abych mohl sledovat výkon a stav aplikace a rychle reagovat na potenciální problémy.
  \item
  Jako plnitel chci v systému vystupovat pod ID, které je náhodně vygenerováno, aby bylo zajištěno mé soukromí.
  \item
  Jako správce dotazníků chci být schopen upravovat text již existující otázky, abych mohl opravit překlepy nebo vylepšit formulaci otázky.
  \item
  Jako plnitel chci měnit heslo svého účtu, abych zajistil bezpečnost svého účtu.
\end{itemize}

\subsection{Volitelné}\label{subsec:volitelne}

\begin{itemize}
  \item
  Jako plnitel chci dostávat e-mail upozornění na nové úkoly a na nesplněné úkoly, abych byl vždy informován o svých úkolech a termínech.
  \item
  Jako plnitel chci dostávat push notifikace v prohlížeči na nové úkoly a na nesplněné úkoly, abych byl vždy aktuálně informován o svých úkolech.
\end{itemize}

\section{Stakeholdeři a jejich zájmy}\label{sec:stakeholderi}

\subsection*{Uživatelé aplikace}\label{subsec:uzivatele-aplikace}

\begin{itemize}
  \item
  Podpora mobilních zařízení
\end{itemize}

\subsubsection*{Plnitelé (klienti a pacienti)}\label{subsubsec:plnitele}

\begin{itemize}
  \item
  Jednoduché rozhraní pro plnění úkolů, které zajistí zlepšení jejich zdravotního stavu a poskytne terapeutům potřebné informace pro jejich léčbu.
\end{itemize}

\subsubsection*{Terapeuti}\label{subsubsec:terapeuti}

\begin{itemize}
  \item
  Jednoduché rozhraní vhodné pro uživatele s nízkými technickými znalostmi.
  \item
  Jednoduchý přístup k výsledkům pacientů s přehlednou vizuální reprezentací.
  \item
  Spolupráce s plniteli v rámci terapie formou domácích úkolů pro zlepšení poskytované péče a prevenci kriminality.
\end{itemize}

\subsection*{Ostatní}\label{subsec:ostatni}

\subsubsection*{Vývojáři}\label{subsubsec:vyvojari}

\begin{itemize}
  \item
  Dobře dokumentovaný kód a vhodně dokumentovaná architektura aplikace.
  \item
  Dobře nastavené procesy pro prevenci chyb (testy, kontinuální integrace, apod.).
  \item
  Jednoduchá rozšiřitelnost aplikace.
  \item
  Vysoká modifikovatelnost aplikace.
\end{itemize}

\subsubsection*{Provozovatel (správce) software}\label{subsubsec:provozovatel-spravce-software}

\begin{itemize}
  \item
  Jednoduché nasazení aplikace na server.
  \item
  Schopnost monitorování aplikace.
\end{itemize}

\subsubsection*{Výzkumníci}\label{subsubsec:vyzkumnici}

\begin{itemize}
  \item
  Možnost tvorby komplexních výzkumných studií.
  \item
  Možnost pokročilé analýzy dat vlastním nástrojem.
\end{itemize}

\subsubsection*{Vedení organizace}\label{subsubsec:vedeni-organizace}

\begin{itemize}
  \item
  Nízká cena za vývoj a provoz aplikace.
  \item
  Nízké časové nároky na zaučení plnitel.
  \item
  Vysoké zabezpečení aplikace.
\end{itemize}
