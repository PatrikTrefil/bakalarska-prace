\chapter{Analýza požadavků}\label{ch:analyza-pozadavku}

Výsledkem prvních několika schůzek byla následující neformální specifikace.
Již v tomto stádiu bylo jasné, že bude potřeba definovat společný jazyk pro komunikaci mezi všemi zúčastněnými stranami.
Dalším krokem bylo tedy vytvoření doménového modelu, který dostal formu UML diagramu a textového souboru definic entit.
Časem byla neformální specifikace převedena do formy user stories.
Tato konverze byla provedena zejména pro lepší organizaci vývoje softwaru.
Dále jsme identifikovali stakeholdery a sepsali jejich zájmy.
Všechny tyto artefakty se nachází v následujících podkapitolách.

\section{Specifikace}\label{sec:specifikace}

\subsection{Skupiny uživatelů}\label{subsec:skupiny-uzivatelu}

\begin{itemize}
\item
  Uživatel = pacient/klient

\item
  Zaměstnanci NUDZ

  \begin{itemize}
  \item
    Správce dotazníků = terapeut/výzkumník
  \item
    Zadavatel dotazníků = terapeut/výzkumník
  \end{itemize}
\item
  Technická podpora
\end{itemize}


\subsection{Funkční požadavky}\label{subsec:funkcni-pozadavky}
\begin{itemize}
\item
  Zadavatel/správce dotazníků může zadávat dotazníky a úkoly konkrétnímu uživateli, nastavit frekvenci opakování a nastavit start/deadline.

  \begin{itemize}
  \item
    Start je čas, kdy lze dotazník nejdříve vyplnit.
  \item
    Deadline je čas, kdy lze dotazník nejpozději vyplnit.
  \item
    Dotazník se skládá z otázek, které mohou přijímat odpovědi těchto typů:

    \begin{itemize}
    \item
      text
    \item
      více možností (možno zvolit právě jeden)
    \item
      více možností (možno zvolit libovolný počet)
    \end{itemize}
  \item
    u každé otázky je možno nastavit podmíněné zobrazení, tj.\ otázka se uživateli zobrazí pokud je splněna podmínka definovaná správcem dotazníku při vytvoření.
  \item
    V aplikaci by mělo být možné vytvořit \hyperref[sec:ukazkovyDotaznik]{ukázkový dotazník}.
  \end{itemize}
\item
  Správce dotazníků je schopen vytvářet, mazat dotazníky a způsoby vyhodnocení (\emph{v dotazníku lze upravit pouze již existující otázky a lze v nich upravit obsah otázky a obsah již existujících odpovědí; \textbf{nelze} přidat/odebrat otázky, přidat/odebrat možné odpovědo, ani upravit způsob výpočtu skóre - slouží pouze pro opravy/objasnění otázek}).

  \begin{itemize}
  \item
    Při tvorbě dotazníku je možno nastavit automaticky počítané metriky pomocí vzorce (např.\ \texttt{otázka1\ +\ otázka2\ -\ otázka4}, kde \texttt{otázka1}, \texttt{otázka2} a \texttt{otázka4} jsou proměnné reprezentující výsledky příslušných otázek).
  \end{itemize}
\item
  Uživatel může v aplikaci vypracovat dotazníky a úkoly, které mu byly přiděleny správcem/zadavatelem dotazníků.

  \begin{itemize}
  \item
    Uživatel je schopen vyplnit část dotazníku, uložit si dosud zodpovězené otázky a v budoucnu vyplňování dotazníku dokončit.
  \end{itemize}
\item
  Uživatel může smazat svá data.
\item
  Správce/zadavatel dotazníků je schopen vidět výsledky dotazníků a úkolů všech uživatelů.
  Správce/zadavatel může vybrat data, která budou vizualizována na grafech.
\item
  Správce dotazníků/člen technické podpory je schopen vytvářet/mazat uživatelské účty, účty pro ostatní správce a účty pro zadavatele dotazníků.
\item
  Správce dotazníků/člen technické podpory je schopen měnit přístupová práva všech ostatních účtů.
\item
  Uživatel je schopen měnit heslo svého účtu a je schopen svůj účet smazat.
\item
  Výsledky dotazníků uživatelů může správce/zadavatel dotazníků z aplikace exportovat do formátu CSV\@.
\item
  Aplikace bude pouze v Českém jazyce, ale bude připravena na internacionalizaci.
\item
  Uživatel v systému vystupuje pod ID, které je náhodně vygenerováno.
\end{itemize}


\subsubsection{Viditelnost vyhodnocení dotazníků}

\begin{itemize}
\item
  Uživatel je schopen vidět vyhodnocení svých dotazníků, u kterých to správce/zadavatel dovolil.
\item
  Správce/zadavatel dotazníků může dovolit uživateli vidět výsledky konkrétního dotazníku - číselná hodnota/graf/text.
\item
  Rozhodnutí o viditelnosti výsledků pro uživatele dělá správce/zadavatel dotazníku při zadávání dotazníku.
\item
  Viditelnost výsledků lze nastavit následujícími způsoby:

  \begin{itemize}
  \item
    uživatel vidí výsledky
  \item
    uživatel nevidí výsledky
  \item
    uživatel vidí výsledky pouze výsledek dotazníku splňuje podmínku (např.\ výsledné skóre je větší než 10)
  \end{itemize}
\end{itemize}


\subsubsection{Volitelné}

\begin{itemize}
\item
  Sledování jak dlouho trvalo odpovědět na konkrétní otázku, kdy uživatel dotazník vyplnil, počet změn odpovědí.
\item
  E-mail upozornění na nové úkoly, na nesplněné úkoly.
\item
  Push notifikace v prohlížeči na nové úkoly, na nesplněné úkoly.
\item
  Chat mezi terapeutem a uživatelem.
\item
  Rozšíření již vystavěné infrastruktury pro zadávání dotazníků pro zadávání cvičení (např.\ přečíst článek, zhlédnout video nebo vypracovat interaktivní cvičení).
\end{itemize}


\subsection{Nefunkční požadavky}\label{subsec:nefunkcni-pozadavky}

\begin{itemize}
\item
  Uživatelské rozhraní by mělo být vhodné pro uživatelé s malými technickými znalostmi.
\item
  K programu by měla být dodána uživatelská a technická dokumentace.
\end{itemize}


\subsection{Deployment}\label{subsec:deployment}

\begin{itemize}
\item
  Aplikace bude spouštěna v kontejneru.
\item
  Parametry serveru: (lze navýšit v případě potřeby)

  \begin{itemize}
  \item
    CPU: 2 jádra
  \item
    RAM: 4 GB
  \item
    HDD: volných 29 GB
  \end{itemize}
\end{itemize}


\subsection{Monitoring}\label{subsec:monitoring}

\begin{itemize}
\item
  Bude k dispozici rozhraní pro monitorování aplikace.
\end{itemize}

\section{Doménový model}\label{sec:domenovy-model}

\includesvg[width=0.9\textwidth]{diagrams/domainModel}

\begin{tcolorbox}
Dědičnost v UML vyjadřuje jiný vztah než v programování.
Třída je množina a odvozená třída je její podmnožina.
Vztahy mezi odvozenými třídami, tedy podmnožinami, mohou být 4 typů podle toho zda-li se podmnožiny překrývají a zda-li je jejich sjednocení množina definovaná třídou, od které odvozujeme.
Tento vztah je vždy popsán v poznámce k třídě, od které odvozujeme.
\end{tcolorbox}

\subsection{Definice entit}\label{subsec:definice-entit}

\begin{description}
\item[Osoba]
  Lidská bytost, která má vztah k systému.

\item[Zaměstnanec]
  Osoba, která je zaměstnána v NUDZ\@.

\item[SprávceÚkolů]
  Zaměstnanec, který je zodpovědný za správu úkolů v systému.

\item[ZadavatelÚkolů]
  Zaměstnanec, který zadává pro plnitelé úkoly v systému.

\item[Plnitel]
  Osoba, která vypracovává úkoly v systému.

\item[Klient]
  Plnitel, který je klientem NUDZ (platí za poskytovanou péči).

\item[Pacient]
  Plnitel, který je pacientem NUDZ (poskytovaná péče je hrazena pojišťovnou).

\item[DefiniceÚkol]
  Obecný potenciálně znovupoužitelný popis činnosti plnitele.

\item[DefiniceFormuláře]
  Definice úkolu, která obsahuje formulář, který je určen k vyplnění plnitelem.

\item[ZadáníÚkolu]
  Zadání úkolu pro konkrétního plnitele na základě definice úkolu.

\item[VypracováníÚkolu]
  Vypracování úkolu plnitelem na základě zadání úkolu.

\item[VypracováníFormuláře]
  Vypracování úkolu ve tvaru vyplnění formuláře.

\item[AnalýzaVypracováníFormuláře]
  Analýza vypracování formuláře, která je určena k uložení odvozených dat.

\item[NedokončenéVyplněníFormuláře]
  Nedokončené vyplnění formuláře, které slouží k uložení částečně vyplněných formulářů.
  (Umožňuje uživateli přerušit vyplňování formuláře a pokračovat později)

\end{description}

\subsection{Omezení}\label{subsec:omezeni}

Zde jsou vypsána omezení, která nejsou vyjádřena v diagramu.
Každé omezení je napsáno jak v neformální textové podobě, tak v \href{https://www.omg.org/spec/OCL/2.4/About-OCL}{Object constraint language}.

\subsubsection{NedokončenéVyplněníFormuláře}

Nedokončené vyplnění formuláře pro definici formuláře, zadání úkolu a uživatele může existovat pouze pokud je zadání úkolu pro uživatele a zadání úkolu zadává stejný formulář, který je částečně vyplněn nedokončeným vyplněním formuláře.

\begin{verbatim}
context plnitel: Plnitel inv
  plnitel->vlastni->forAll(
      nedokonceneVyplneniFormulare |
          nedokonceneVyplneniFormulare
            ->castecneVypracovava
            ->zadanPro = self
          and
          nedokonceneVyplneniFormulare
            ->castecneVypracovava
            ->definovano = nedokonceneVyplneniFormulare->vyplnuje
          )
\end{verbatim}

\subsubsection{ZadáníÚkolu}

Definice úkolu musí logicky odpovídat vypracování úkolu.
Např.\ nemůžeme považovat přečtení článku jako vypracování úkolu, který je definován jako vypracování formuláře.

\begin{verbatim}
context z: ZadáníÚkolu inv
    if vypracování.oclIsKindOf(VypracováníFormuláře) then
        definováno.oclIsKidnOf(DefiniceFormuláře)
    endif
\end{verbatim}

\subsubsection{VypracováníÚkolu}

ID musí být unikátní.

\begin{verbatim}
context v1, v2: VypracováníÚkolu inv v1.ID = v2.ID implies v1 = v2
\end{verbatim}

\subsubsection{DefiniceÚkolu}

ID musí být unikátní.

\begin{verbatim}
context d1, d2: DefiniceÚkolu inv d1.ID = d2.ID implies d1 = d2
\end{verbatim}

\subsubsection{Osoba}

ID musí být unikátní.

\begin{verbatim}
context o1, o2: Osoba inv o1.ID = o2.ID implies o1 = o2
\end{verbatim}

\section{User stories}\label{sec:user-stories}

Zde je seznam funkčních požadavků ve standardizované formě.
Jedná se o přepis neformální specifikace ze sekce~\ref{sec:specifikace}.

\begin{itemize}
  \item
  Jako zaměstnanec chci mít možnost založit účet pro klienta/pacienta, abych mohl využít systém v rámci terapie/výzkumu.
  \item
  Jako klient/pacient chci aby moje data nemohli číst ostatní klienti/pacienti, protože jsou soukromá.
  \item
  Jako terapeut/výzkumník chci aby byl klient/pacient schopen plnit pouze úkoly, které mu byly zadány, aby se zachovala integrita sbíraných dat.
  \item
  Jako klient/pacient chci být schopen zobrazit svá data, abych věděl co je o mě v systému evidováno (např.\ vyplněné dotazníky).
  \item
  Jako správce dotazníků chci mít možnost zadávat dotazníky a úkoly konkrétnímu uživateli, nastavit frekvenci opakování a nastavit start/deadline, abych mohl správně koordinovat procesy a aktivity v systému.
  \item
  Jako správce dotazníků chci mít možnost vytvářet otázky v dotaznících, které mohou přijímat odpovědi typu text, více možností (možno zvolit právě jeden) a více možností (možno zvolit libovolný počet), abych mohl vytvářet různorodé a komplexní dotazníky.
  \item
  Jako správce dotazníků chci mít možnost nastavit podmíněné zobrazení otázky, takže otázka se uživateli zobrazí pokud je splněna podmínka definovaná mnou při vytvoření, což umožní flexibilní a přizpůsobené průzkumy.
  \item
  Jako správce dotazníků chci mít možnost vytvářet, mazat dotazníky a způsoby vyhodnocení, abych mohl udržovat aktuální a relevantní soubor dotazníků.
  \item
  Jako správce dotazníků chci mít možnost nastavit automaticky počítané metriky pomocí vzorce (např.\ \texttt{otázka1 + otázka2 - otázka4}), což mi umožní efektivně vyhodnocovat odpovědi.
  \item
  Jako uživatel chci mít možnost vypracovat dotazníky a úkoly, které mi byly přiděleny správcem/zadavatelem dotazníků, abych mohl aktivně participovat v systému.
  \item
  Jako uživatel chci mít možnost vyplnit část dotazníku, uložit si dosud zodpovězené otázky a v budoucnu vyplňování dotazníku dokončit, aby bylo vyplňování dotazníků flexibilní a pohodlné.
  \item
  Jako uživatel chci mít možnost smazat svá data, abych měl kontrolu nad svými daty.
  \item
  Jako správce/zadavatel dotazníků chci mít možnost vidět výsledky dotazníků a úkolů všech uživatelů a vybrat data, která budou vizualizována na grafech, abych mohl monitorovat a analyzovat výsledky.
  \item
  Jako správce dotazníků/člen technické podpory chci mít možnost vytvářet/mazat uživatelské účty, účty pro ostatní správce a účty pro zadavatele dotazníků, abych mohl efektivně spravovat uživatelské účty.
  \item
  Jako správce/zadavatel dotazníků chci mít možnost exportovat výsledky dotazníků uživatelů z aplikace do formátu CSV, abych mohl dělat pokročilé analýzy dat.
  \item
  Jako uživatel chci, aby aplikace byla v Českém jazyce, ale byla připravena na internacionalizaci, abych mohl používat aplikaci v jazyce, kterému rozumím.
  \item
  Jako uživatel chci vidět vyhodnocení svých dotazníků, u kterých to správce/zadavatel dovolil, abych mohl sledovat svůj pokrok a výsledky.
  \item
  Jako správce/zadavatel dotazníků chci moci dovolit uživateli vidět výsledky konkrétního dotazníku - číselná hodnota/graf/text, abych mohl sdílet výsledky a zpětnou vazbu s uživateli.
  \item
  Jako uživatel chci, aby uživatelské rozhraní bylo vhodné pro uživatele s malými technickými znalostmi, abych mohl snadno používat aplikaci bez ohledu na své technické dovednosti.
  \item
  Jako uživatel chci, aby k programu byla dodána uživatelská a technická dokumentace, abych mohl lépe porozumět, jak aplikace funguje a jak ji používat.
  \item
  Jako technický pracovník chci, aby aplikace byla spouštěna v kontejneru, abych mohl snadno spravovat a nasazovat aplikaci.
  \item
  Jako technický pracovník chci mít k dispozici rozhraní pro monitorování aplikace, abych mohl sledovat výkon a stav aplikace a rychle reagovat na potenciální problémy.
  \item
  Jako uživatel chci v systému vystupovat pod ID, které je náhodně vygenerováno, aby bylo zajištěno mé soukromí.
  \item
  Jako správce dotazníků chci být schopen upravovat text již existující otázky, abych mohl opravit překlepy nebo vylepšit formulaci otázky.
  \item
  Jako uživatel chci měnit heslo svého účtu, abych zajistil bezpečnost svého účtu.
\end{itemize}

\subsection{Volitelné}\label{subsec:volitelnuxe9}

\begin{itemize}
  \item
  Jako uživatel chci dostávat e-mail upozornění na nové úkoly a na nesplněné úkoly, abych byl vždy informován o svých úkolech a termínech.
  \item
  Jako uživatel chci dostávat push notifikace v prohlížeči na nové úkoly a na nesplněné úkoly, abych byl vždy aktuálně informován o svých úkolech.
\end{itemize}

\section{Stakeholdeři a jejich zájmy}\label{sec:stakeholderi}

\subsection{Uživatelé aplikace}\label{subsec:uzivatele-aplikace}

\begin{itemize}
  \item
  Podpora mobilních zařízení
\end{itemize}

\subsubsection{Klienti/pacienti}\label{subsubsec:klientipacienti}

\begin{itemize}
  \item
  Jednoduché rozhraní pro plnění úkolů, které zajistí zlepšení jejich
  zdravotního stavu a poskytne terapeutům potřebné informace pro jejich
  léčbu.
\end{itemize}

\subsubsection{Terapeuti}\label{subsubsec:terapeuti}

\begin{itemize}
  \item
  Jednoduché rozhraní vhodné pro uživatele s nízkými technickými
  znalostmi
  \item
  Jednoduchý přístup k výsledkům pacientů s přehlednou vizuální
  reprezentací
  \item
  Spolupráce s klienty/pacienty v rámci terapie formou domácích úkolů
  pro zlepšení poskytované péče a prevenci kriminality
\end{itemize}

\subsection{Ostatní}\label{subsec:ostatni}

\subsection{Vývojáři}\label{subsec:vyvojari}

\begin{itemize}
  \item
  Dobře dokumentovaný kód a vhodně dokumentovaná architektura aplikace
  \item
  Dobře nastavené procesy pro prevenci chyb (testy, kontinuální
  integrace, \ldots{})
  \item
  Jednoduchá rozšiřitelnost aplikace
  \item
  Vysoká modifikovatelnost aplikace
\end{itemize}

\subsection{Provozovatel (správce) software}\label{subsec:provozovatel-spravce-software}

\begin{itemize}
  \item
  Jednoduché nasazení aplikace na server
  \item
  Monitoring aplikace
\end{itemize}

\subsection{Výzkumníci}\label{subsec:vyzkumnici}

\begin{itemize}
  \item
  Možnost tvorby komplexních výzkumných studií
  \item
  Možnost pokročilé analýzy dat vlastním nástrojem
\end{itemize}

\subsection{Vedení organizace}\label{subsec:vedeni-organizace}

\begin{itemize}
  \item
  Nízká cena za vývoj a provoz aplikace
  \item
  Nízké časové nároky na zaučení uživatelů
  \item
  Vysoké zabezpečení aplikace
\end{itemize}
