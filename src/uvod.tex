\chapter*{Úvod}
\addcontentsline{toc}{chapter}{Úvod}

Tato bakalářská práce vznikla z motivace zlepšit kvalitu poskytované zdravotní péče v oblasti duševního zdraví za pomoci digitálních technologií.
Prvním cílem práce je posílení spolupráce mezi terapeuty a jejich pacienty/klienty.
Druhým cílem práce je zefektivnění sběru dat v oblasti duševního zdraví.
Psychoterapie a výzkum jsou oblasti, které mohou získat velký prospěch z digitalizace, ale vyžadují specializované nástroje.
V této práci navrhneme, implementujeme a otestujeme platformu pro spolupráci terapeutů/výzkumníků a pacientů/klientů/účastníků studie.

Tato práce vznikla ve spolupráci s \href{https://www.nudz.cz/}{Národním ústavem duševního zdraví}.
Tato instituce se zabývá výzkumem v oblasti věd o mozku a chování, realizující sběr a zpracování demografických, epidemiologických a sociologických dat souvisejících s duševním zdravím.
Ústav současně poskytuje i psychiatrickou péči nemocným (\href{https://cs.wikipedia.org/wiki/Národní_ústav_duševního_zdraví}{zdroj}).

Sběr dat o účastnících studie a spolupráce mezi terapeuty a pacienty/klienty může mít mnoho podob.
V této práci se zaměříme na sběr dat pomocí dotazníků, jelikož je to jeden z nejčastějších způsobů sběru dat a lze dobře převést do digitální podoby.

Sekundární využití aplikace je zadávání domácích úkolů pro pacienty/klienty.
Tyto domácí úkoly mohou mít podobu různých mentálních cvičení či vzdělávacích aktivit jako čtení článků nebo sledování videí.
Tuto funkcionalitu nebudeme implementovat, ale software navrhneme tak, aby ji bylo možno v budoucnu jednoduše přidat.

\section*{Struktura práce}\label{sec:struktura-prace}

Nejprve provedeme analýzu požadavků Národního ústavu duševního zdraví a popíšeme doménu.
Následně prozkoumáme již existující software, který v práci použijeme.
Ná základě průzkumu navrhneme architekturu softwaru.
Poté popíšeme jakým způsobem jsme postupovali při implementaci a proč jsme tento postup zvolili.
Na závěr provedeme akceptační testování a zhodnotíme úspěšnost projektu.