\chapter*{Úvod}
\addcontentsline{toc}{chapter}{Úvod}

Cílem bakalářské práce je navržení, vytvoření a praktické otestování v provozu platformy pro spolupráci terapeutů/výzkumníků a pacientů/klientů/účastníků studie.
Platforma bude sloužit primárně pro sběr dat týkajících se mentálního zdraví uživatelů za účelem zvýšení efektivity poskytované psychoterapie, výzkumu a prevenci kriminality. 
Platformu však bude možno použít i v jiných kontextech.
Sběr dat bude možno provádět jednorázově či dlouhodobě.
Aplikace bude poskytovat základní přehled, vizualizace dat a export dat do formátu CSV.
Tato práce vznikla ve spolupráci s Národním ústavem duševního zdraví.

Sběr dat o pacientech/klientech/účastnících studie může mít mnoho podob.
V této práci se zaměříme na sběr dat pomocí dotazníků. 
Dotazníky jsou v psychologii a psychiatrii velmi častým nástrojem pro sběr dat. 
Dotazníky využívané ve vědeckých studiích mohou být poměrně komplexní. 
Důležitým úkolem je tedy poskytnutí uživatelsky přívětivého rozhraní, které umožní definovat i komplexní dotazníky. 
Rohraní by mělo být intuitivní i pro uživatele s nízkým stupněm technické zdatnosti.

Existující řešení pro sběr dat pomocí dotazníků v oblasti mentálního zdraví jsem v době navrhování řešení nenašel. 
Obecná řešení pro sběr dat pomocí dotazníků existují, ale typicky běží pouze v cloudu. 
Národní ústav duševního zdraví vyžaduje plnou kontrolu nad svými daty, takže cloudová řešení nepřipadají v úvahu. 
Již existující řešení také typicky neřeší dlouhodobý sběr dat, ale pouze jednorázové dotazníky. 
Tuto funkcionalitu bude třeba implementovat a integrovat do platformy.

Sekundární využití aplikace je zadávání domácích úkolů pro pacienty/klienty.
Tyto domácí úkoly mohou mít podobu různých mentálních cvičení či vzdělávacích aktivit jako čtení článků nebo sledování videií. 
Tuto funkcionalitu nebudu implementovat, ale software navrhneme tak, aby ji bylo možno v budoucnu jednoduše přidat.

Nejprve provedeme analýzu požadavků Národního ústavu duševního zdraví a vytvoříme doménový model. 
Následně prozkoumáme již existující software, který v práci použijeme. 
Ná základě průzkumu navrhneme architekturu softwaru.
Poté začneme pracovat na implementaci. 
Na závěr provedeme akceptační testování a zhodnotíme úspěšnost projektu. 
Po celou dobu zůstaneme v kontaktu s pracovníky Národního ústavu duševního zdraví a maximálně využijeme jejich znalosti oboru.