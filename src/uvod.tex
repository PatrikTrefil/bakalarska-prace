\chapter*{Úvod}
\addcontentsline{toc}{chapter}{Úvod}

Tato bakalářská práce vznikla z motivace zlepšit kvalitu poskytované zdravotní péče v oblasti duševního zdraví za pomoci digitálních technologií.
Léčba duševních potíží se nazývá psychoterapie a trvá obvykle několik měsíců až let.
V průběhu terapie pacient pravidelně dochází na sezení s terapeutem.
Na těchto sezeních terapeut zadává pacientovi různé úkoly, které pacient následně vypracuje mimo sezení.
Úkoly mají často formu dotazníků, mentálních cvičení nebo vzdělávacích aktivit jako je čtení článků.
Tyto úkoly jsou často zadávány papírovou formou a následně ručně vyhodnocovány.
Digitalizací tohoto procesu lze zajistit efektivnější spolupráci mezi terapeuty a pacienty a automatizovaný sběr dat o pacientech.
Získaná data lze využít pro sledování vývoje stavu pacientů a také pro výzkum.
Cílem je tedy posílení spolupráce mezi terapeuty a jejich pacienty/klienty.
Psychoterapie využívá poznatky z psychologie, což je věda, která studuje lidské chování a mentální procesy.
Výzkum z oblasti psychologie může mít mnoho forem.
Jedna z častých forem výzkumu je sběr dat o skupině pomocí dotazníků.
Dotazníky mají často papírovou formu a následně jsou ručně vyhodnocovány.
Digitalizací tohoto procesu lze zajistit efektivnější sběr dat a automatizované vyhodnocování.
Psychoterapie a výzkum jsou oblasti, které mohou získat velký prospěch z digitalizace, ale vyžadují specializované nástroje.
V této práci navrhneme, implementujeme a otestujeme platformu pro spolupráci terapeutů/výzkumníků a pacientů/klientů/účastníků studie.

Tato práce vznikla ve spolupráci s \href{https://www.nudz.cz/}{Národním ústavem duševního zdraví (NUDZ)}, který poskytl svou expertízu a zkušenosti v oblasti duševního zdraví.
Tato instituce se zabývá výzkumem neurobiologických a psychosociálních mechanismů spojených se vznikem a průběhem nejzávažnějších duševních poruch~\cite{nudz-profil}.
Ústav současně poskytuje i psychiatrickou péči nemocným a plánuje vytvořenou platformu sám používat.

Sběr dat o účastnících studie a spolupráce mezi terapeuty a pacienty/klienty může mít mnoho podob.
My se zaměříme na sběr dat pomocí dotazníků, jelikož je to jeden z nejčastějších způsobů sběru dat a lze dobře převést do digitální podoby.

Sekundární využití aplikace je zadávání domácích úkolů pro pacienty/klienty.
Tyto domácí úkoly mohou mít podobu různých mentálních cvičení či vzdělávacích aktivit jako čtení článků nebo sledování videí.
Všechny zmíněné aktivity bude možno zadávat a vyhodnocovat v rámci aplikace.

Zbytek práce má následující strukturu.
Na začátku provedeme analýzu požadavků Národního ústavu duševního zdraví a popíšeme doménu (kapitola~\ref{ch:analyza-pozadavku}).
Následně prozkoumáme již existující software, který v práci použijeme (kapitola~\ref{ch:analyza-existujicich-reseni-pro-praci-s-formulari}).
Ná základě průzkumu navrhneme architekturu softwaru.
Poté popíšeme jakým způsobem jsme postupovali při implementaci a proč jsme tento postup zvolili (kapitola~\ref{ch:vyvojova-dokumentace}).
Vysvětlíme, jak se aplikace používá z pohledu uživatele (kapitola~\ref{ch:uzivatelska-dokumentace}) a jak se aplikace spravuje z pohledu administrátora (kapitola~\ref{ch:administratorska-prirucka}).
Dále zhodnotíme jednotlivé části procesu vývoje a zhodnotíme celkovou úspěšnost projektu (kapitola~\nameref{ch:zhodnoceni-vyvoje}).
Na závěr shrneme dosažené výsledky a navrhneme možnosti dalšího vývoje (kapitola~\ref{ch:zaver}).