\chapter{Návrh aplikace}\label{ch:navrh-aplikace}

Tato kapitola popisuje, jak byla aplikace navržena.
První popíšeme jakým způsobem jsme se rozhodli řešit požadavky zadavatele v sekci~\ref{sec:navrh-reseni}.
Poté navrhneme architekturu systému v sekci~\ref{sec:architektura}.
Následně se budeme věnovat návrhu uživatelského rozhraní formou wireframů (sekce~\ref{sec:navrh-uzivatelskeho-rozhrani}).
Následně popíšeme výběr technologií (sekce~\ref{sec:vyber-technologii}) a způsob ukládání dat v systému (sekce~\ref{sec:ukladani-dat}).


\section{Návrh řešení}\label{sec:navrh-reseni}

Jedním z požadavků byla dostupnost aplikace jak na počítačích, tak na mobilních zařízeních.
Zvážili jsme dvě možnosti, jak tohoto dosáhnout.
První možností je vyvinout mobilní aplikaci a desktopovou aplikaci zvlášť.
Druhou možností je vyvinout webovou aplikaci s responzivním rozhraním, která bude dostupná na všech zařízeních.
Jelikož nepotřebujeme žádné nativní funkce zařízení, tak je výhodnější vyvinout pouze jednu webovou aplikaci.
Webová aplikace má také výhodu v tom, že ji není potřeba instalovat a aktualizovat.
Výhodou řešení webovou aplikací je, že zadavatel má zkušenosti s jejich provozem a má existující infrastrukturu.

Řešení webovou aplikací vyžaduje vývoj uživatelského rozhraní a serveru.
Nároky na uživatelské rozhraní jsou v dnešní době velmi vysoké.
Rozhraní musí být rychlé, responzivní a intuitivní.
V případě serveru je potřeba se zaměřit zejména na návrh rozhraní a bezpečnost.
Budeme se snažit použít co nejvíce již existujícího software, jak bylo zadáno v požadavku~\ref{itm:r-nr-4}
Naším cílem bude vybrat software, který nám dá dostatek flexibility a zároveň bude dostatečně stabilní a udržitelný.


\section{Architektura}\label{sec:architektura}

Nyní navrhneme architekturu aplikace.
Budeme využívat architekturu typu klient-server.
Klientem bude webová aplikace, která bude spuštěna ve webovém prohlížeči uživatele.
Serverová část bude mít distribuovanou architekturu, jelikož bude využívat službu Form.io.
Pro její popis použijeme \href{https://c4model.com/}{C4 model} a zaměříme se na 3 úrovně abstrakce - kontext, kontejnery a komponenty.

\subsection{Kontext}\label{subsec:kontext}

Začneme kontextovou vrstvou (Obr.~\ref{fig:architecture-context}), která zobrazuje vztahy mezi naším systémem, jeho uživateli a ostatními systémy.

\begin{figure}[H]
    \centering
    \includegraphics[width=0.9\textwidth]{diagrams/architecture/context}
    \caption{C4 model - Systémový kontext}\label{fig:architecture-context}
\end{figure}

\subsection{Kontejnery}\label{subsec:kontejnery}

Nyní popíšeme kontejnery, které tvoří náš systém, a vztahy mezi nimi (Obr.~\ref{fig:architecture-containers}).
Aplikace byla rozdělena na kontejnery primárně na základě logické souvislosti funkcionalit, které poskytují.
U většiny kontejnerů je zřejmý jejich účel.
Některé však vyžadují bližší popis, který se nachází níže.

\begin{figure}[H]
    \centering
    \includegraphics[width=0.9\textwidth]{diagrams/architecture/containers}
    \caption{C4 model - Kontejnery}\label{fig:architecture-containers}
\end{figure}

\begin{tcolorbox}
    Pokud je vztah mezi dvěma kontejnery vyznačený jako gateway, znamená to, že komunikace probíhá skrze prostředníka, který poskytuje vrstvu abstrakce.
    Gateway snižuje provázanost komunikujících systémů (coupling) a zajišťuje modifikovatelnost.
    Bližší informace o tomto vzoru lze nalézt v \href{https://martinfowler.com/articles/gateway-pattern.html}{článku Gateway} od Martin Fowler.
\end{tcolorbox}

\begin{tcolorbox}
    Monitoring server monitoruje všechny ostatní kontejnery.
    Relace byly vynechány pro zachování čitelnosti diagramu.
\end{tcolorbox}

\subsubsection{Správa formulářů}\label{subsubsec:sprava-formularu}

Tento kontejner poběží na serveru a bude umožňovat CRUD\footnotemark{} operace nad definicemi formulářů.
Také bude poskytovat rozhraní pro sběr dat z formulářů a jejich export.

\subsubsection{Správa úkolů}\label{subsubsec:sprava-ukolu}

Tento kontejner poběží na serveru a bude umožňovat CRUD\footnotemark[\value{footnote}] operace nad úkoly.

\footnotetext{CRUD je zkratka pro \textit{create, read, update, delete}, což jsou základní operace při práci s daty~\cite{crud-def}}

\subsubsection{Správa nedokočených vyplnění}\label{subsubsec:sprava-nedokoncenych-vyplneni}

Tento kontejner se bude starat o ukládání nedokončených vyplnění formulářů.
Tento kontejner bude ve své podstatě fungovat jako uložiště typu klíč-hodnota, kde klíčem je identifikátor formuláře a hodnotou jsou nedokončená vyplnění.
Pro zachování jednoduchosti systému nebude kontejner posílat požadavky na kontejner pro správu formulářů.
Kontejner tedy nebude validovat existenci formuláře pro který je nedokončené vyplnění ukládáno ani nebude kontrolovat, zda-li nedokončené vyplnění obsahuje validní data.
Přístup ke všem operacím, které tento kontejner poskytuje, dostanou pouze plnitelé úkolů, což je skupina uživatelů definovaná v požadavku~\ref{itm:r-nr-1}.
Způsob ukládání dat, které vlastní tento kontejner, je popsán v sekci~\ref{subsec:sprava-dat-o-nedokoncenych-vyplneni-formularu}.

\subsubsection{Export dat}\label{subsubsec:export-dat}

Pro export sesbíraných dat z aplikace potřebujeme získat data o úkolech a odevzdání formulářů.
Každý úkol spojíme s odevzdáním, které bylo vytvořeno v rámci splnění tohoto úkolu.
Data o úkolech a odevzdání však vlastní dvě různé služby.
Spojení dat můžeme provést na klientovi a nebo na serveru.
Obě varianty jsou možné.
Umístění procesu spojení na server zajistí lepší interoperabilitu našeho systému s ostatními systémy a umožní programově řídit export dat z aplikace.

Pro tyto účely byl vytvořen speciální kontejner jehož úkolem je získat data z ostatních služeb a spojit je.
Jelikož export dat nebude probíhat příliš často a sesbíraná data budou poměrně malá (očekáváme řádově maximálně stovky záznamů), není nutné dbát na vysokou výkonnost této služby.

Máme několik možností implementace této služby.
První možností je použití návrhového vzoru \href{https://microservices.io/patterns/data/api-composition.html}{API composition}.
Tento vzor pracuje se spojením uvnitř operační paměti.
Implementace tohoto vzoru je velmi jednoduchá, ale spojení dat může být neefektivní.
Druhou možností je použití návrhového vzoru \href{https://microservices.io/patterns/data/cqrs.html}{Command query responsibility segregation}.
Tento vzor využívá repliky dat z obou datových zdrojů.
Tato možnost je efektivnější, ale složitější na implementaci.

Z vlastností obou možností je vidět, že API composition je vhodnější řešení našeho problému.
Finální implementace byla realizována vytvoření tRPC routeru v rámci NextJS aplikace (Obr.~\ref{fig:deployment}).

\subsection{Komponenty}\label{subsec:komponenty}

Na komponenty budeme dělit pouze kontejner \textit{Webová aplikace}.
Tento kontejner rozdělíme na zaměstnaneckou a uživatelskou sekci (Obr.~\ref{fig:architecture-component-web-application}).

\begin{figure}[H]
    \centering
    \includegraphics[width=0.9\textwidth]{diagrams/architecture/webApplicationComponent}
    \caption{C4 model - Komponenta webová aplikace}\label{fig:architecture-component-web-application}
\end{figure}


\section{Nasazení}\label{sec:deployment}

Na obr.~\ref{fig:deployment} je znázorněn diagram nasazení pomocí notace \href{https://c4model.com/#DeploymentDiagram}{C4 modelu}.
Diagram obsahuje i použité technologie.
Odůvodnění výběru těchto technologií je popsáno v sekci~\ref{sec:vyber-technologii}.
Všechny komponenty jsou nasazeny na jediný server, jak bylo požadováno v zadání v sekci~\ref{subsec:deployment}.
Pro snížení počtu nasazených Docker kontejnerů byly kontejnery Databáze nedokončených vyplnění a Databáze úkolů nasazeny na jednu instanci databáze.
Implementace dovoluje i nasazení těchto komponent na různé instance databáze.

\begin{figure}[H]
    \centering
    \includegraphics[width=0.95\textwidth]{diagrams/deployment}
    \caption{Diagram nasazení}\label{fig:deployment}
\end{figure}


\section{Návrh uživatelského rozhraní}\label{sec:navrh-uzivatelskeho-rozhrani}

Jak již víme z kapitoly~\ref{ch:analyza-pozadavku} o analýze požadavků, budeme potřebovat navrhnout rozhraní pro plnitele úkolů a také pro terapeuty.
Předpokládáme, že všichni uživatelé aplikace mají nízké technické dovednosti a proto se budeme snažit navrhnout co nejjednodušší rozhraní.
Abychom předešli vývoji rozhraní, které by zadavatel nepovažoval za vhodné, tak vytvoříme wireframy, které necháme zadavatelem schválit.
Výsledné wireframy naleznete v přílohách práce~\ref{sec:wireframy-uzivatelskeho-rozhrani}.


\section{Výběr technologií}\label{sec:vyber-technologii}

V této sekci se budeme zabývat výběrem konkrétních technologií, které budeme používat při vývoji aplikace.
Chceme volit populární a vyspělé nástroje, jelikož programátoři, kteří budou aplikaci v budoucnu rozšiřovat, je pravděpodobně budou znát.
Zároveň také chceme co nejméně komplikovat workflow a nasazení, abychom zbytečně nezvyšovali nároky na programátory, kteří budou aplikaci v budoucnu rozšiřovat.
Budeme vybírat primárně z open-source řešení.

\subsection{Software pro práci s formuláři}\label{subsec:software-pro-praci-s-formulari}

Software pro práci s formuláři je klíčovou součástí aplikace jak popsáno v kapitole~\ref{ch:analyza-existujicich-reseni-pro-praci-s-formulari}.
Problematika správy formulářů je detailně popsána v sekci~\ref{sec:problematika-spravy-formularu}.
V závěru kapitoly~\ref{ch:analyza-existujicich-reseni-pro-praci-s-formulari} jsme se rozhodli pro využití software Form.io.
Nyní popíšeme, jak přesně tento software bude využit.

Jádro software Form.io je open-source a má licenci Open Software License 3.0, která dovoluje komerční použití.
Jádro také obsahuje základní systém pro správu uživatelů.
Tato bezplatná část má samozřejmě jistá omezení oproti placené verzi.
Nevýhodou je, že použití vlastního autentizačního poskytovatele je dostupné pouze v placených verzích.
Navzdory omezením bezplatné verze využijeme jádro software Form.io pro správu formulářů.
Do budoucna se nabízí přechod na placenou verzi.

V době učinění tohoto rozhodnutí bylo předpokládáno, že funkce pro uložení nedokončených vyplnění je součástí jádra.
Ukázalo se, že přestože tato funkce není v dokumentaci označená jako placená, tak je dostupná pouze v placené verzi.
Dojmu, že se jedná o bezplatnou funkce, napomáhá i to, že v bezplatném rozhraní pro tvorbu formulářů existují tlačítka pro uložení nedokončených vyplnění.
Až poté, co se ji uživatel pokusí použít, zjistí že jsou nefunkční.
Z těchto důvodů bylo nutné implementovat vlastní řešení pro uložení nedokončených vyplnění.

\subsection{Frontend aplikace}\label{subsec:frontend-aplikace}

Frontend aplikace je klientská část aplikace běžící v prohlížeči.
Hlavní problémy této části aplikace pro nás budou rychlost, responzivita a design.
V dnešní době existuje mnoho knihoven a frameworků, které řeší tyto problémy.
Níže jsou popsány zvolené technologie, knihovny a odůvodnění jejich použití.

\begin{description}
    \item[\href{https://getbootstrap.com/}{Bootstrap}]
    Knihovna, která zjednodušuje stylizaci webových aplikací.
    Jedná se o dlouho existující, vyspělou a populární knihovnu.
    Alternativy jako \href{https://tailwindcss.com/}{Tailwind CSS} nabízejí větší míru flexibility, ale neposkytují žádné hotové komponenty.
    Tento projekt nemá za cíl vytvářet vlastní designovou identitu, proto je výhodnější použít hotové komponenty, které Bootstrap nabízí.
    \item[JavaScript/\href{https://www.typescriptlang.org/}{TypeScript}]
    JavaScript je výchozí programovací jazyk pro vývoj webových aplikací.
    Alternativní řešení jsou v současné době omezená.
    TypeScript je nadstavba JavaScript poskytující statickou typovou kontrolu, což pomůže předejít mnoha chybám.
    Přestože TypeScript zkomplikuje a zpomalí vývoj, tak jej použijeme pro prevenci chyb.
    \item[\href{https://react.dev/}{React}]
    React je knihovna používaná mimo jiné pro vývoj webových aplikací.
    Tento nástroj zajišťuje reaktivní aktualizaci uživatelského rozhraní na základě změn dat aplikaci.
    React je dle průzkumu \href{https://2023.stateofjs.com/en-US/libraries/front-end-frameworks/}{State of JavaScript} z roku 2022 nejpoužívanější front-end framework.
    Jedná se o vyspělý framework, který má velkou komunitu a mnoho existujících knihoven.
    \item[\href{https://nextjs.org/}{NextJS}]
    Místo volby jednotlivých balíků pro základy řešení routing, middleware, sdílených layoutů apod.\ zvolme framework, který nám tyto funkce poskytne.
    NextJS je meta-framework postavený na nástroji React, který rozšiřuje jeho funkce.
    Jedná se o full-stack framework, takže jej využijme i na serverovou část.
    \item[\href{https://react-bootstrap.github.io/}{react-bootstrap}]
    Tato knihovna značně zjednodušuje použití knihovny Bootstrap v React aplikacích a navíc zajišťuje alespoň základní webovou přístupnost.
    \item[\href{https://www.npmjs.com/package/react-i18next}{react-i18next}]
    Aplikace bude podporovat více jazyků a proto je vhodné použít knihovnu, která nám usnadní práci s překlady.
    Oproti alternativám jako \href{https://github.com/airbnb/polyglot.js}{Polyglot.js} nebo \href{https://github.com/formatjs/formatjs}{Format.js} nabízí knihovna react-i18next více funkcí.
    Knihovna react-i18next je s 2.9 milióny stažení za týden nejpopulárnější ze zvažovaných knihoven dle počtu stažení z npm registry~\cite{react-i18next-npm}.
    \item[\href{https://github.com/formio/react}{React formio}]
    Knihovna pro vykreslování formulářů na základě schématu.
    Tato oficiální knihovna je součástí projektu Form.io.
    Tato knihovna také poskytuje podporu pro autentizaci uživatele pomocí Form.io serveru.
    Tuto funkci však nebudeme používat z důvodů popsaných v bodě o \href{itm:next-auth}{NextAuth.js}.
    \item[\href{https://next-auth.js.org/}{NextAuth.js}]\label{itm:next-auth}
    Knihovna pro autentizaci uživatelů v prohlížeči i na serveru.
    Nástroj pracuje skvěle s NextJS a lze jeho konfigurace je relativně jednoduchá.
    Ukázalo se, že knihovna Formio React není vhodná pro autentizaci uživatele.
    První důvod je, že vyžaduje, aby si vývojář psal vlastní logiku pro ochranu stránek, přesměrování apod.
    To zbytečně vytváří prostor pro chyby.
    Druhým důvodem je, že inicializace autentizace se dělá pouze na straně klienta, jelikož knihovna nepodporuje server-side rendering.
    To má značně negativní vliv na výkon aplikace.
    Třetí důvod je špatná dokumentace knihovny.
    Poslední výhodou použití jiné knihovny než Formio React je vytvoření vrstvy abstrakce v kódu nad autetifikačním systému.
    Díky tomuto je možné kdykoliv vyměnit autentizační systém za jiný.
    Jako alternativy jsem zvažoval \href{https://www.passportjs.org/}{Passport.js} a \href{https://next-auth.js.org/}{NextAuth.js}.
    Passport.js je více nízko-úrovňový a opět vyžaduje, aby si vývojář psal vlastní logiku.
    Oproti Formio React má však pěknou dokumentaci.
    Nakonec jsem se ale přiklonil k NextAuth.js.
    \item[\href{https://zod.dev/}{Zod}]
    Pro validaci dat na serveru a validaci formulářů na klientovi využijeme validační knihovnu.
    Knihovnu Zod jsem zvolil jelikož má rozhraní, se kterým se dobře pracuje.
    \item[\href{https://react-hook-form.com/}{React hook form}]
    Aplikace bude obsahovat formuláře pro přihlášení, správu uživatelů, tvorbu úkolů a mnoho dalších.
    Formuláře lze tvořit pomocí přímé interakce s prvky HTML dokumentu.
    S rozhraními těchto prvků se však špatně pracuje například protože používají pro všechny hodnoty typ \texttt{string}.
    Pro lepší práci s formuláři využijeme knihovnu React hook form.
    Díky této knihovně budeme moci snadno validovat formuláře, získávat hodnoty formulářů se správnými typy a zobrazovat chybové hlášky.
    Lze zvážit také knihovnu \href{https://formik.org}{Formik}.
    Recenze na internetu byly stejně dobré jako pro React hook form, ale integrace s validační knihovnou Zod existuje pouze jako \href{https://github.com/robertLichtnow/zod-formik-adapter}{komunitní balíček}.
    React hook form má oficiální integraci s validační knihovnou Zod pomocí \href{https://github.com/react-hook-form/resolvers}{balíčku resolvers}.
    \item[\href{https://tanstack.com/query/latest}{Tanstack query}]
    Aplikace bude obsahovat mnoho dat, která budou načítána ze serveru.
    Pro zvýšení kvality kódu a zlepšení výkonu aplikace využijeme knihovnu.
    Populární řešení jsou Tanstack query a \href{https://swr.vercel.app/}{SWR}\@.
    Tanstack query oproti SWR nabízí více funkcí a mnoho z nich nám značně ulehčí práci.
    Mezi tyto funkce patří například vývojové nástroje.
    Tanstack query má \href{https://tanstack.com/query/v4/docs/react/devtools}{oficiální vývojové nástroje} s širokou funkcionalitou.
    Vývojové nástroje pro SWR existují pouze jako \href{https://github.com/koba04/swr-devtools}{komunitní balíček}, který má pouze omezenou funkcionalitu.
    \item[\href{https://tanstack.com/table/v8}{Tanstack table}]
    Aplikace bude obsahovat velké množství tabulek.
    Naše tabulky budou podporovat filtrování, řazení a stránkování.
    Toto jsou běžné funkce, na které již existuje spoustu řešení.
    Vzhledem k tomu, že používáme knihovnu Tanstack query, tak se nabízí použít i knihovnu Tanstack table.
    Výhodou této knihovny je, že je velmi flexibilní a umožňuje vytvářet vlastní komponenty pro vykreslování tabulek.
    Jedná se o tzv. \textit{headless UI} knihovnu, což nám umožní definovat vlastní vzhled tabulek.
    \item[\href{https://recharts.org/en-US/}{Recharts}]
    Aplikace bude potřebovat vizualizovat sesbíraná data a proto potřebujeme knihovnu na vykreslování grafů.
    Pro tyto účely byly zváženy knihovny \href{https://github.com/airbnb/visx}{visx} od AirBnB, \href{https://github.com/apache/echarts} od Apache a \href{https://github.com/recharts/recharts}{Recharts}.
    Knihovna visx má velmi komplikovanou dokumentaci i příklady.
    Tento nástroj má evidentně strmou výukovou křivku a pro naše účely je zbytečně komplexní.
    Knihovna echarts je velmi populární, ale nemá oficiální podporu pro React.
    Knihovna Recharts má kvalitní dokumentaci obsahující jak jednoduché tak složitější příklady a navíc má přímou podporu pro React.
\end{description}

\subsection{Middleware}\label{subsec:middleware}

Komunikaci mezi klientem a serverem s sebou nese celou řadu problémů.
Musíme vyřešit jakým způsobem budeme data přenášet, jak data budeme serializovat\slash deserializovat a jak budeme data validovat.
Všechny tyto problémy řeší knihovny z kategorie middleware.
Použití takové knihovny nám navíc zajistí typovou bezpečnost a zlepší vývojářskou zkušenost.
Abychom neztratili interoperabilitu serveru s ostatními aplikacemi, použijeme knihovnu, která umožní vytvořit i REST API\@.
Mezi populární volby řešení patří knihovny \href{https://grpc.io/}{gRPC} od firmy Google, \href{https://www.npmjs.com/package/json-rpc-2.0}{json-rpc-2.0} implementující \href{https://www.jsonrpc.org/specification}{standard JSON-RPC 2.0}  a \href{https://trpc.io/}{tRPC}.
Knihovna gRPC však nefunguje v prohlížeči~\cite{state-of-grpc}, takže ji pro tento projekt nemůžeme použít.
Knihovna \href{https://trpc.io/}{tRPC} má skvělou podporu pro TypeScript a je kompatibilní s knihovnou NextAuth.js.
Knihovna tRPC je oproti knihovně json-rpc-2.0 značně populárnější, nabízí více funkcí a má aktivnější vývoj.
Proto jsem se nakonec rozhodl použít knihovnu tRPC\@.

\subsection{Backend aplikace}\label{subsec:backend-aplikace}

\begin{description}
    \item[\href{https://nginx.org/}{nginx}]
    Veškeré požadavky na server budou směrovány přes reverse proxy.
    To nám umožní konfigurovat routing, SSL certifikáty apod.
    Mezi zvážené možnosti patří \href{https://httpd.apache.org/}{Apache HTTP Server} a nginx.
    Nginx má přehlednější dokumentaci a čitelnější formát konfigurace.
    Nginx má navíc největší podíl z počítačů na internetu~\cite{server-survey}.
    \item[NextJS]
    Důvody popsány v sekci~\ref{subsec:frontend-aplikace}.
    \item[Form.io]
    Důvody popsány v sekci~\ref{subsec:software-pro-praci-s-formulari}.
    \item[\href{https://www.mongodb.com/}{MongoDB}]
    Form.io server podporuje pouze MongoDB pro ukládání dat\@.
    \item[\href{https://www.postgresql.org/}{PostgreSQL}]
    Potřebujeme databázi pro ukládání dat o úkolech a také nedokončené odpovědi na dotazníky.
    Chceme řešení, které je dostatečně stabilní a jeho licence umožňuje komerční použití.
    Výběr konkrétního řešení není příliš důležitý z následujících důvodů.
    Naše nároky na databázový systém jsou poměrně nízké a navíc budeme používat abstrakci nad databází, která nám umožní databázový systém kdykoliv vyměnit.
    PostgreSQL patří mezi nejlepší řešení splňující všechny naše požadavky.
    \item[\href{https://www.prisma.io/}{Prisma}]
    Object relational mapper (ORM) je software používaný pro překlad mezi objektovou reprezentací dat a reprezentací, kterou používají databázové systémy~\cite{orm-definition}.
    Mapování mezi objekty a tabulkami relací je problém, který je těžké vyřešit obecně, a proto ORM knihovny často generují neoptimální SQL dotazy.
    Tyto dotazy mohou více či méně celou aplikaci zpomalovat.
    Tyto knihovny však umožňují zrychlit vývoj aplikace a zjednodušit práci s databází.
    Díky tomuto nástroji se můžeme vyhnout psaní velkého množství \textit{boilerplate} kódu a soustředit se na implementaci business logiky.
    Mezi populární volby patří knihovny \href{https://github.com/drizzle-team/drizzle-orm}{Drizzle ORM}, \href{https://github.com/typeorm/typeorm}{TypeORM} a \href{https://github.com/prisma/prisma}{Prisma}.
    Knihovna Prisma je nejvyspělejší a nejpopulárnější z těchto knihoven co se týče počtu stažení za týden z npm registry.
    Knihovna je navíc vyvíjen komerční firmou, což zvyšuje její dlouhodobou udržitelnost.
\end{description}


\section{Ukládání dat}\label{sec:ukladani-dat}

Tato sekce se věnuje způsobu ukládání dat v aplikaci.
V podsekci~\ref{subsec:logicky-datovy-model} popíšeme logický datový model.
Následně navrhneme řešení správy dat o nedokončených vyplnění formuláři v podsekci~\ref{subsec:sprava-dat-o-nedokoncenych-vyplneni-formularu}.

\subsection{Logický datový model}\label{subsec:logicky-datovy-model}

Nyní popíšeme jak data budete ukládat na logické úrovni.
Logický datový model je zobrazen na obr.~\ref{fig:logical-data-model} a používá notaci \href{https://www.omg.org/spec/UML/2.5.1/PDF}{UML}.
Nebudeme se zabývat detaily ukládání dat o uživatelích a formulářích, jelikož o to se stará software Form.io.
Entity, jejichž reprezentací se nechceme zabývat, jsou označeny na obrázku tmavě šedou barvou.
Budeme se zabývat ukládáním úkolů a částečných vyplnění formulářů.
Zvolíme relační datový model pro modelování těchto dat.
Třídy v diagramu budou reprezentovat tabulky v databázi.

Pro stav úkolu budeme potřebovat místo tradičních dvou stavů (dokončený a nedokončený) stavy tři.
Důvody k tomuto rozhodnutí jsou blíže popsány v sekci~\ref{sec:propojeni-spravy-ukolu-a-spravy-formularu}.

\begin{figure}[H]
    \centering
    \includegraphics[width=0.9\textwidth]{diagrams/logicalDataModel}
    \caption{Logický datový model}\label{fig:logical-data-model}
\end{figure}

\subsection{Správa dat o nedokončených vyplnění formulářů}\label{subsec:sprava-dat-o-nedokoncenych-vyplneni-formularu}

Plnitel si může vytvořit koncept pro libovolný dotazník.
Při odevzdání dotazníku informuje systém spravující formuláře automaticky pomocí webhook systému systém spravující nedokončené vyplnění formulářů.
Tento proces je zobrazen na obr.~\ref{fig:drafts-management} jako diagram aktivity v notaci \href{https://www.omg.org/spec/UML/2.5.1/PDF}{UML}.
Díky tomu nezůstávají nedokončené vyplnění formulářů v systému navždy.
Pokud však tento webhook selže, nedokončené vyplnění formuláře v systému zůstane.
Tato situace je velmi nepravděpodobná, takže ji nebudeme řešit.
Pokud by se v průběhu užívání ukázalo, že se jedná o větší problém než se předpokládalo, lze zavést omezení na životnost konceptu.
Například bychom mohli ukládat pouze koncepty, které byly použity v posledních 30 dnech.
Tím bychom zajistili, že i pokud webhook selže, tak se koncept v systému nezachová déle než 30 dní.

\begin{figure}[H]
    \centering
    \includegraphics[width=0.9\textwidth]{diagrams/draftsManagement}
    \caption{Správa nedokončených vyplnění dotazníků}\label{fig:drafts-management}
\end{figure}
